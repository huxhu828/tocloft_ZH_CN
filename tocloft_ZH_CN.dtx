% \iffalse meta-comment
%
% tocloft.dtx
% Author: Peter Wilson, Herries Press
% Maintainer: Will Robertson (will dot robertson at latex-project dot org)
% Copyright 1998-2003 Peter R. Wilson
% Copyright 2009-2017 Will Robertson
%
% This work may be distributed and/or modified under the
% conditions of the LaTeX Project Public License, either
% version 1.3c of this license or (at your option) any
% later version. <http://www.latex-project.org/lppl.txt>
%
% This work has the LPPL maintenance status "maintained".
% The Current Maintainer of this work is Will Robertson.
%
% This work consists of the files listed in the README file.
%
%
%<*driver>
\documentclass{ltxdoc}
\usepackage{hypdoc}
\EnableCrossrefs
\CodelineIndex
\setcounter{StandardModuleDepth}{1}

\usepackage{graphicx}
\usepackage[T1]{fontenc}
\usepackage{tabularray}
\usepackage{caption}
\usepackage{fontspec}
\usepackage{float} %%%% 防止表格等浮动

%%%%%%%%%%%%% 以下设置中文字体 %%%%%%%%%%%%%%%%%%%%%%%%%%%%%%%%%%%%%%%%%
\usepackage{xeCJK}  %%

\setCJKfamilyfont{Heiti}{Source Han Sans Regular} %%%% 自定义\Heiti命令,显示思源黑体,用于标题页标题的中文部分
\newcommand{\Heiti}{\CJKfamily{Heiti}} %%%% 自定义\Heiti命令,显示思源黑体,用于标题页标题的中文部分

\setCJKfamilyfont{heiti}{Source Han Sans Light} %%自定义\heiti命令,显示思源黑体,用于正文的章节标题
\newcommand{\heiti}{\CJKfamily{heiti}} %%自定义\heiti命令,显示思源黑体,用于正文的章节标题

\setCJKfamilyfont{songti}{思源宋体 CN Light}  %%%% 自定义\songti命令,显示思源宋体,用于正文
\newcommand{\songti}{\CJKfamily{songti}} %%%% 自定义\songti命令,显示思源宋体,用于正文

\setCJKmainfont{思源宋体 CN Light} %%%% 设置中文的主字体为思源宋体 CN Light
%\setmainfont{Source Serif Pro} %%%% 设置英文的主字体为Source Serif Pro,也可以设为Times New Roman

\setCJKfamilyfont{kaiti}{KaiTi} %%设置中文字体楷体,用于强调
\newcommand{\kaiti}{\CJKfamily{kaiti}} %%设置中文字体楷体,用于强调
%%%%%%%%%%%%% 以上设置中文字体 %%%%%%%%%%%%%%%%%%%%%%%%%%%%%%%%%%%%%%%%%

%%%%%%%%%%%%% 以下设置中文版式 %%%%%%%%%%%%%%%%%%%%%%%%%%%%%%%%%%%%%%%%%
\usepackage{indentfirst} %%% 首行缩进
\setlength{\parindent}{2em} %%% 缩进2个字符(中文为2个字)
\linespread{1.5} %%% 设置行间距
%%%%%%%%%%%%% 以上设置中文版式 %%%%%%%%%%%%%%%%%%%%%%%%%%%%%%%%%%%%%%%%%
\usepackage{changepage} %%%用于整体缩进,\begin{adjustwidth}{2cm}{1cm}

%%%%%%% 以下在 tabular 表格中定制 横线如\hlinew{1.2pt} %%%%%%
\makeatletter
\def\hlinew#1{%
\noalign{\ifnum0=`}\fi\hrule \@height #1 \futurelet
\reserved@a\@xhline}
\makeatother%
%%%%%%% 以上在 tabular 表格中定制 横线如\hlinew{1.2pt} %%%%%%

%%%%%%% 以下自定义脚注 %%%%%%%%%%%%%%%%%%%%%%%%%%%%%%%%%%%%
\setlength{\footnotesep}{0.5cm} %%%设置几第脚注之间的距离
\setlength{\skip\footins}{2.0em} %%%设置脚注与正文之间的距离
%%\renewcommand\footnoterule{} %%%定义脚注线为空
\renewcommand\footnoterule{
     \kern -3pt                         % This -3 is negative
     \hrule width 0.6\textwidth height 0.6pt % of the sum of this 1
     \kern 2pt} %%%
%%%%%%% 以上自定义脚注 %%%%%%%%%%%%%%%%%%%%%%%%%%%%%%%%%%%%

%%%%%%% 以下制作索引 %%%%%%%%%%%%%%%%%%%%%%%%%%%%%%%%%%%%
\usepackage {makeidx}
\makeindex
%%%%%%% 以上制作索引 %%%%%%%%%%%%%%%%%%%%%%%%%%%%%%%%%%%%

%%%%%%% 以下调整目录条目之间的间距 %%%%%%%%%%%%%%%%%%%%%%%%%%%%%%%%%%%%
\usepackage{tocloft}
\setlength{\cftbeforesecskip}{14pt} %%% 节(sec)之前的空白为14pt
\renewcommand{\cftsecafterpnum}{\vspace{3pt}} %%% 节(sec)之后的空白为3pt
\setlength{\cftbeforetoctitleskip}{20pt} %%% “目录”二字的段前间距为20pt
\setlength{\cftaftertoctitleskip}{50pt}  %%% “目录”二字的段后间距为50pt
\renewcommand{\cftsubsecafterpnum}{\vspace{3pt}} %%% 小节(subsec)之后的空白为3pt
%%%%%%% 以上调整目录条目之间的间距 %%%%%%%%%%%%%%%%%%%%%%%%%%%%%%%%%%%%



\renewcommand{\contentsname}{\centerline{\heiti {\Large 目\ \ \ 录}}}   %%% 在{document}后面加入该命令,将"contents"变成“目  录”
\renewcommand{\refname}{\heiti 参考文献}
\renewcommand{\figurename}{\Heiti 图}
\renewcommand{\tablename}{\Heiti 表}
\renewcommand{\abstractname}{\heiti {\Large 摘\ 要}}
\renewcommand{\listfigurename}{\centerline{\heiti {\large 图形目录}}}
\renewcommand{\listtablename}{\centerline{\heiti {\large 表格目录}}}

%%%%%%%%%%%% 以下设置书签、目录 %%%%%%%%%%%%%%%%%%%%%%%%%%%%%%%%%%%%%%%%
\usepackage{xcolor}
\usepackage[colorlinks=true,linkcolor=red]{hyperref}
%%%%%%%%%%%% 以上设置书签、目录 %%%%%%%%%%%%%%%%%%%%%%%%%%%%%%%%%%%%%%%%%



\begin{document}
  \DocInput{tocloft_ZH_CN.dtx}
\end{document}
%</driver>
%
%<*usc>
\NeedsTeXFormat{LaTeX2e}
\ProvidesPackage{tocloft}[2017/08/31 v2.3i parameterised ToC, etc., typesetting]
%</usc>
%
% \fi
%
% \changes{v0.1}{1998/12/31}{First public (alpha) release}
% \changes{v0.1a}{1999/01/15}{Improved documentation}
% \changes{v0.2}{1999/01/17}{Uses the stdclsdv package}
% \changes{v0.2a}{1999/01/17}{Interfaced with the tocbibind package}
% \changes{v0.2b}{1999/03/07}{Corrected failure when used with float package}
% \changes{v0.2c}{1999/06/30}{Added raggedright example}
% \changes{v0.3}{1999/08/22}{Added option to prevent ToC head change}
% \changes{v1.0}{1999/09/19}{Added PackageNote reporting}
% \changes{v1.1}{2000/02/11}{Added cftXpresnum commands}
% \changes{v1.1}{2000/02/11}{Added support for subfigure package}
% \changes{v1.1a}{2000/02/11}{Changed X to Z in section 2.2}
% \changes{v2.0}{2001/03/03}{Added command for new list of...}
% \changes{v2.0}{2001/03/03}{Removed requirement for stdclsdv package}
% \changes{v2.0}{2001/03/15}{Removed requirement for hyperref option}
% \changes{v2.1}{2001/04/08}{Added control over parskip in ToC etc.}
% \changes{v2.2}{2001/04/17}{Another fix for hyperref}
% \changes{v2.3}{2002/06/15}{Made to work with the koma classes}
% \changes{v2.3a}{2002/10/03}{Partial fix for Part changes}
% \changes{v2.3b}{2003/01/20}{More fixes for Part changes}
% \changes{v2.3c}{2003/09/26}{More fixes for hyperref}
% \changes{v2.3d}{2009/09/04}{New maintainer and tiny bug fix}
% \changes{v2.3e}{2010/10/13}{Remove a message from the console output}
%
% \def\dtxfile{\texttt{tocloft.dtx}}
% \def\fileversion{v1.1a} \def\filedate{2000/02/11}
% \def\fileversion{v2.0}  \def\filedate{2001/03/15}
% \def\fileversion{v2.1}  \def\filedate{2001/04/08}
% \def\fileversion{v2.2}  \def\filedate{2001/04/17}
% \def\fileversion{v2.3}  \def\filedate{2002/06/15}
% \def\fileversion{v2.3a} \def\filedate{2002/10/03}
% \def\fileversion{v2.3b} \def\filedate{2003/01/20}
% \def\fileversion{v2.3c} \def\filedate{2003/09/26}
% \def\fileversion{v2.3d} \def\filedate{2009/09/04}
% \def\fileversion{v2.3e} \def\filedate{2010/10/13}
% \def\fileversion{v2.3f} \def\filedate{2013/05/02}
% \def\fileversion{v2.3g} \def\filedate{2014/06/14}
% \def\fileversion{v2.3h} \def\filedate{2017/08/29}
% \def\fileversion{v2.3i} \def\filedate{2017/08/31}
% \def\fileversion{v2.3j} \def\filedate{2020/01/10}
%
% \DoNotIndex{\',\.,\@M,\@@input,\@addtoreset,\@arabic,\@badmath}
% \DoNotIndex{\@centercr,\@cite}
% \DoNotIndex{\@dotsep,\@empty,\@float,\@gobble,\@gobbletwo,\@ignoretrue}
% \DoNotIndex{\@input,\@ixpt,\@m}
% \DoNotIndex{\@minus,\@mkboth,\@ne,\@nil,\@nomath,\@plus,\@set@topoint}
% \DoNotIndex{\@tempboxa,\@tempcnta,\@tempdima,\@tempdimb}
% \DoNotIndex{\@tempswafalse,\@tempswatrue,\@viipt,\@viiipt,\@vipt}
% \DoNotIndex{\@vpt,\@warning,\@xiipt,\@xipt,\@xivpt,\@xpt,\@xviipt}
% \DoNotIndex{\@xxpt,\@xxvpt,\\,\ ,\addpenalty,\addtolength,\addvspace}
% \DoNotIndex{\advance,\Alph,\alph}
% \DoNotIndex{\arabic,\ast,\begin,\begingroup,\bfseries,\bgroup,\box}
% \DoNotIndex{\bullet}
% \DoNotIndex{\cdot,\cite,\CodelineIndex,\cr,\day,\DeclareOption}
% \DoNotIndex{\def,\DisableCrossrefs,\divide,\DocInput,\documentclass}
% \DoNotIndex{\DoNotIndex,\egroup,\ifdim,\else,\fi,\em,\endtrivlist}
% \DoNotIndex{\EnableCrossrefs,\end,\end@dblfloat,\end@float,\endgroup}
% \DoNotIndex{\endlist,\everycr,\everypar,\ExecuteOptions,\expandafter}
% \DoNotIndex{\fbox}
% \DoNotIndex{\filedate,\filename,\fileversion,\fontsize,\framebox,\gdef}
% \DoNotIndex{\global,\halign,\hangindent,\hbox,\hfil,\hfill,\hrule}
% \DoNotIndex{\hsize,\hskip,\hspace,\hss,\if@tempswa,\ifcase,\or,\fi,\fi}
% \DoNotIndex{\ifhmode,\ifvmode,\ifnum,\iftrue,\ifx,\fi,\fi,\fi,\fi,\fi}
% \DoNotIndex{\input}
% \DoNotIndex{\jobname,\kern,\leavevmode,\let,\leftmark}
% \DoNotIndex{\list,\llap,\long,\m@ne,\m@th,\mark,\markboth,\markright}
% \DoNotIndex{\month,\newcommand,\newcounter,\newenvironment}
% \DoNotIndex{\NeedsTeXFormat,\newdimen}
% \DoNotIndex{\newlength,\newpage,\nobreak,\noindent,\null,\number}
% \DoNotIndex{\numberline,\OldMakeindex,\OnlyDescription,\p@}
% \DoNotIndex{\pagestyle,\par,\paragraph,\paragraphmark,\parfillskip}
% \DoNotIndex{\penalty,\PrintChanges,\PrintIndex,\ProcessOptions}
% \DoNotIndex{\protect,\ProvidesClass,\raggedbottom,\raggedright}
% \DoNotIndex{\refstepcounter,\relax,\renewcommand,\reset@font}
% \DoNotIndex{\rightmargin,\rightmark,\rightskip,\rlap,\rmfamily,\roman}
% \DoNotIndex{\roman,\secdef,\selectfont,\setbox,\setcounter,\setlength}
% \DoNotIndex{\settowidth,\sfcode,\skip,\sloppy,\slshape,\space}
% \DoNotIndex{\symbol,\the,\trivlist,\typeout,\tw@,\undefined,\uppercase}
% \DoNotIndex{\usecounter,\usefont,\usepackage,\vfil,\vfill,\viiipt}
% \DoNotIndex{\viipt,\vipt,\vskip,\vspace}
% \DoNotIndex{\wd,\xiipt,\year,\z@}
%
% \newcommand*{\Lpack}[1]{\textsf {#1}}           ^^A typeset a package
% \newcommand*{\Lopt}[1]{\textsf {#1}}            ^^A typeset an option
% \newcommand*{\file}[1]{\texttt {#1}}            ^^A typeset a file
% \newcommand*{\Lcount}[1]{\textsl {\small#1}}    ^^A typeset a counter
% \newcommand*{\pstyle}[1]{\textsl {#1}}          ^^A typeset a pagestyle
% \newcommand*{\Lenv}[1]{\texttt {#1}}            ^^A typeset an environment
% \providecommand{\bs}{\textbackslash}
%
%
% \title{{\Huge\Lpack{tocloft}}\ {\huge \Heiti 宏包}\ \thanks{此文件(\dtxfile)的版
% 本号为 \fileversion{},最后修订日期为 \filedate{}。}}
% \author{
%   作者:Peter Wilson,Herries Press \\
%   维护:Will Robertson \\
%   翻译:黄旭华\ \thanks{赣南医学院第一附属医院神经科的一名 \LaTeX{}\ 业余爱好者。}
% }
%%%% \date{\filedate}
% \date{2020\,/\,1\,/\,10}
% \maketitle
%
% \vspace{8cm}
% \begin{abstract}
% \Lpack{tocloft}\ 宏包提供了控制目录(table of Contents)、图目录(list of Figures)和
% 表目录(list of Tables)的排版设计的方法。可以定义新类型的“\ldots\ 目录”。
%
% 该宏包已使用 \Lpack{tocbibind}、\Lpack{minitoc}、\Lpack{ccaption}、\Lpack{subfigure}、
% \Lpack{float}、\Lpack{fncychap}\ 和 \Lpack{hyperref}\ 等宏包进行了测试。
% \end{abstract}
% \phantomsection
\addcontentsline{toc}{section}{\heiti 摘要}
% \clearpage
% \tableofcontents
% \makeatletter \renewcommand{\@dotsep}{9.0} \makeatother
% \listoffigures
% \listoftables
% \StopEventually{}
%
%
% \newpage
% \section{\heiti 介绍}
%
% 在标准类(standard classes)中,目录(Table of Contents,ToC)、图目录(List of Figures,LoF)和
% 表目录(List of Tables,LoT)的排版设计(typographic design)是固定的,或者更准确地说,
% 它被隐藏在类定义(class definitions)中。\Lpack{tocloft}\ 宏包为作者提供了修改设计
% 以满足特定文档需求的句柄(handles)。
%
% 宏包的元素(elements)是作为 ISO 标准~\cite{PRW96i}\ 排版的类(class)和包(package)的
% 一部分开发的。本手册根据 \LaTeX{} \textsc{docstrip}\ 实用程序(utility)的约定(conventions)排版,
% 该实用程序可自动提取 \LaTeX{}\ 宏源文件(macro source files)~\cite{GOOSSENS94}。
%
% 第~\ref{sec:usc}~节描述了该宏包的使用。该宏包源代码的注释见第~\ref{sec:code}~节。
%
% 该宏包已至少与 \Lpack{tocbibind}\ 宏包~\cite{TOCBIBIND}、\Lpack{minitoc}\ 宏
% 包~\cite{MINITOC}、\Lpack{ccaption}\ 宏包~\cite{CCAPTION}、\Lpack{subfigure}\ 宏
% 包~\cite{SUBFIGURE} (2.0 版和 2.1 版)、\Lpack{algorithm}\ 宏
% 包~\cite{ALGORITHM} (依次称为 \Lpack{float}\ 宏包~\cite{FLOAT})和 \Lpack{fncychap}\ 宏
% 包~\cite{FNCYCHAP} 结合进行了测试。它也适用于 \Lpack{hyperref}\ 宏包。请给我任何
% 关于您认为该宏包可以如何改进的意见,或者关于你如何使用它的任何有趣的例子。
% \footnote{感谢卡斯珀(Kasper,\texttt{kbg@dkik.dk})、罗兰(Rowland,\texttt{rebecca@astrid.u-net.com})、
% 约翰·福斯特(John Foster,\texttt{john@isjf.demon.co.uk})、
% 利·内夫(Lee Nave,\texttt{nave@math.washington.edu})、
% 安德鲁·瑟伯(Andrew Thurber,\texttt{athurber@emba.uvm.edu})等的建议。}
%
% \subsection[\LaTeX 的方法]{\LaTeX {\heiti 的方法}}
%
% 这是对 \LaTeX{}\ 如何处理目录(Table of Contents,ToC)的一般描述(general description)。
% 由于图目录(List of Figures)和表目录(List of Tables)的处理类似,我将在不失一般性的情况下讨论 ToC。
%
% \DescribeMacro{\addcontentsline}
% 如果文档(document)包含 |\tableofcontents| 命令,\LaTeX{}\ 将生成一个 \file{.toc}\ 文件。
% 分节命令(sectioning commands)\ \footnote{对于图形(figures)和表格(tables),分节命令则是 \texttt{\bs caption}\ 命令,
% 该命令填充(populates) \file{.lof}\ 和 \file{.lot}\ 文件。}\ 通过调
% 用 \LaTeX{}\ 的 \\[4pt] {\large \color{blue}{|\addcontentsline{|\meta{file}|}{|\meta{kind}|}{|\meta{title}|}|}} \\[4pt] 命令将条目(entries)放
% 入 \file{.toc}\ 文件中,这里的 \meta{file}\ 是文件扩展名(例如,|toc|),\meta{kind}\ 是条目类型
% (例如,|section| 或 |subsection|),\meta{title}\ 是(已编号的[numberered])标题文本(title text)。
% 在有数字的情况下,\meta{title}\ 参数以 |{\numberline{number} title-text}| 的形式给出。
%
% {\kaiti 注意}:\Lpack{hyperref}\ 宏包不喜欢作者使用 |\addcontentsline|。要使其与 \Lpack{hyperref}\ 一起正常工作,
% 通常必须在 |\addcontentsline| 之前立即放置 |\phantomsection| (在 \Lpack{hyperref}\ 宏包中定义的宏)。
%
% \DescribeMacro{\contentsline}
% |\addcontentsline| 命令以\\[4pt]
% {\large \color{blue}{|\contentsline{|\meta{kind}|}{|\meta{title}|}{|\meta{page}|}|}}\\[4pt]
% 的形式向给定文件写入一个条目(entry),这里的 \meta{page}\ 是页码(page number)。
% 对于每个 \meta{kind},\LaTeX{}\ 提供一个命令 |\l@kind{|\meta{title}|}{|\meta{page}|}|,
% 它执行 |\contentsline| 条目(entry)的实际排版(actual typesetting)。
%
% \newcommand{\maxx}{120}       ^^A picture width
% \newcommand{\maxxm}{118}      ^^A \maxx - 2\
% \newcommand{\maxy}{55}        ^^A picture height
% \newcommand{\maxym}{53}       ^^A \maxy - 2
% \newcommand{\findent}{20}     ^^A indent
% \newcommand{\findentp}{22}    ^^A \findent + 2
% \newcommand{\fnumwidth}{10}   ^^A numwidth
% \newcommand{\ftocrmarg}{30}   ^^A \@tocrmarg
% \newcommand{\fpnumwidth}{20}  ^^A \@pnumwidth
% \newcommand{\fipn}{30}        ^^A \findent + \fnumwidth
% \newcommand{\frmarg}{90}      ^^A \maxx - \ftocrmarg
% \newcommand{\frnum}{100}      ^^A \maxx - \fpnumwidth
% \newcommand{\fyi}{10}         ^^A 1st y height
% \newcommand{\fyim}{8}         ^^A \fyi - 2
% \newcommand{\fyii}{20}        ^^A 2nd y height
% \newcommand{\fyiii}{25}       ^^A 3rd y height
% \newcommand{\fyiv}{30}        ^^A 4th y height
% \newcommand{\fyv}{40}         ^^A 5th y height
% \newcommand{\fyvp}{42}        ^^A \fyv + 2
% \newcommand{\flin}{4}         ^^A length of leader lines
% \newcommand{\frmargm}{89}     ^^A \frmarg (90) - a little bit
%
% \providecommand{\bs}{\textbackslash}
% \begin{figure}[H]
% \centering
% \setlength{\unitlength}{1mm}
% \begin{picture}(\maxx,\maxy)
%     ^^A side lines and linewidth
%   \put(0,0){\line(0,1){\maxy}}
%   \put(\maxx,0){\line(0,1){\maxy}}
%   \put(0,\maxy){\vector(1,0){\maxx}}
%   \put(2,\maxym){\makebox(0,0)[tl]{\texttt{\bs linewidth\ {\footnotesize(当前行的宽度)}}}}
%     ^^A \@pnumwidth
%   \put(\maxx,\fyi){\vector(-1,0){\fpnumwidth}}
%   \put(\maxxm,\fyim){\makebox(0,0)[tr]{\texttt{\scriptsize \bs @pnumwidth}\ {\kaiti {\tiny (页码宽度)}}}}
%   \put(\frnum,\fyi){\line(0,1){\flin}}
%     ^^A \@tocrmarg
%   \put(\maxx,\fyv){\vector(-1,0){\ftocrmarg}}
%   \put(\maxxm,\fyvp){\makebox(0,0)[br]{\texttt{\scriptsize \bs @tocrmarg}\ {\kaiti {\tiny (标题文本的右边距缩进量)}}}}
%   \put(\frmarg,\fyv){\line(0,-1){\flin}}
%     ^^A indent
%   \put(0,\fyv){\vector(1,0){\findent}}
%   \put(2,\fyvp){\makebox(0,0)[bl]{\textit{\scriptsize indent}\ {\kaiti {\tiny (缩进)}}}}
%   \put(\findent,\fyv){\line(0,-1){\flin}}
%     ^^A numwidth
%   \put(\findent,\fyv){\vector(1,0){\fnumwidth}}
%   \put(\findentp,\fyvp){\makebox(0,0)[bl]{\textit{\scriptsize numwidth}\ {\kaiti {\tiny (编号宽度)}}}}
%   \put(\fipn,\fyv){\line(0,-1){\flin}}
%     ^^A last title line
%   \put(\maxx,\fyii){\makebox(0,0)[br]{487}}
%   \put(\fipn,\fyii){title end\ {\scriptsize(标题结束)}}
%     ^^A second title line
%   \put(\fipn,\fyiii){continue\ {\scriptsize(继续)}\ldots}
%   \put(\frmarg,\fyiii){\makebox(0,0)[br]{\ldots title\ {\scriptsize(标题)}}}
%     ^^A first title line
%   \put(\findent,\fyiv){\textbf{3.5}}
%   \put(\fipn,\fyiv){Heading\ {\scriptsize(标题)}\ldots}
%   \put(\frmarg,\fyiv){\makebox(0,0)[br]{\ldots title\ {\scriptsize(标题)}}}
%     ^^A dotted leader
%   \multiput(\frmargm,\fyii)(-\flin,0){8}{.}
%   \multiput(\frmarg,\fyi)(-\flin,0){2}{\line(0,1){\flin}}
%   \put(\frmarg,\fyi){\vector(-1,0){\flin}}
%   \put(\frmarg,\fyi){\vector(1,0){0}}
%   \put(\frmarg,\fyim){\makebox(0,0)[tr]{\texttt{\scriptsize \bs @dotsep}\ {\kaiti {\tiny (点间距)}}}}
%
% \end{picture}
% \setlength{\unitlength}{1pt}
% \caption[\Heiti 目录、图目录、表目录等的条目布局]{\Heiti 目录(ToC)、图目录(LoF)、表目录(LoT)等的条目布局} \label{fig:ltoc}
% \end{figure}
%
%
% 排版条目(typeset entry)的总体布局(general layout)如图~\ref{fig:ltoc}~所示。排版中使用了
% 三个内部(internal) \LaTeX{}\ 命令。页码(page number)在一个宽度为 |\@pnumwidth| 的盒子
% 右对齐排版(typeset flushright),
% \DescribeMacro{\@pnumwidth}
% 盒子位于右边距(righthand margin)。如果页码太长而无法放入盒子中,
% 则页码将突出到右边距(righthand margin)。
% \DescribeMacro{\@tocrmarg}
% 标题文本(title text)从右边距缩进了 |\@tocrmarg| 给出的量(amount)。
% 请注意,|\@tocrmarg| 应大于 |\@pnumwidth|。有些条目是在标题文本末尾和右边距缩
% 进(righthand margin indentation)之间用点引线(dotted leader)排版的。引线中的点(dots)之间的
% 距离(以{\kaiti 数学单位}\footnote{这里是 18mu 到 1em。}表示)由 |\@dotsep| 的值给出。
% \DescribeMacro{\@dotsep}
% 在标准类(standard classes)中,目录(ToC)、图目录(LoF)、表目录(LoT)等使用相同的值。
%
%    这些内部命令(internal commands)的标准值(standard values)为:
% \begin{itemize}
% \item |\@pnumwidth| = 1.55em
% \item |\@tocrmarg| = 2.55em
% \item |\@dotsep| = 4.5
% \end{itemize}
% 尽管前两个值看起来都是长度(lengths),但是可以通过使用 |\renewcommand| 命令来更改这些值。
%
% 点引线(dotted leaders)不适用于 Part 和 Chapter 的目录条目(ToC entries)
% (也不适用于 \Lpack{article}\ 类及其派生物[derivatives]中的 Section 条目)。
%
% \DescribeMacro{\numberline}
% 每个 |\l@kind| 宏负责设置从左边距(lefthand margin)开始的常规{\kaiti 缩进}(general \textit{indent})
% 和{\kaiti 编号宽度}(\textit{numwidth})。|\numberline{|\meta{number}|}| 宏负责在一个宽度
% 为 \textit{numwidth}\ 的盒子中以左对齐的(flushleft)方式排版数字。如果数字对于盒子来说太长,
% 那么它将突出到(protrude into)标题文本(title text)中。标题文本从左边距(lefthand margin)缩进
% (\textit{indent + numwidth})。也就是说,标题文本是按宽度为
% (|\linewidth| - \textit{indent} - \textit{numwidth} - |\@tocrmarg|)
% 的块(block)来排版的。
%
% \begin{table}[H]
% \begin{center}
% \caption[\Heiti 缩进和编号宽度]{\Heiti 缩进(indents)和编号宽度(numwidths)(单位:\ em)} \label{tab:indents}
% \vspace{-0.4em}
% \begin{tabular}{rcrrrr}
% \hlinew{1.2pt}
% Entry & Level & \multicolumn{2}{c}{Chaptered} & \multicolumn{2}{c}{Otherwise} \\
%       &       & indent & numwidth & indent & numwidth \\ \hlinew{0.7pt}
% part          & -1 & 0    & --- & 0    & --- \\
% chapter       & 0  & 0    & 1.5 &      &     \\
% section       & 1  & 1.5  & 2.3 & 0    & 1.5 \\
% subsection    & 2  & 3.8  & 3.2 & 1.5  & 2.3 \\
% subsubsection & 3  & 7.0  & 4.1 & 3.8  & 3.2 \\
% paragraph     & 4  & 10.0 & 5.0 & 7.0  & 4.1 \\
% subparagraph  & 5  & 12.0 & 6.0 & 10.0 & 5.0 \\
% figure/table  & (1) & 1.5 & 2.3 & 1.5  & 2.3 \\ \hlinew{1.2pt}
% \end{tabular}
% \end{center}
% \end{table}
%
% 上面的表~\ref{tab:indents}~列出了{\kaiti 缩进}(\textit{indent})和{\kaiti 编号宽度}(\textit{numwidth})的标准值(standard values)。
% 对于部分(part)没有明确的编号宽度(\textit{numwidth});而是在编号(number)和标题文本(title text)之间留出 1em 的间隙(gap)。
% 请注意,对于分节命令(sectioning command),值取决于文档类(document class)是否提供 |\chapter| 命令。
% 此外,这让我有些惊讶,表格(table)和图形(figure)的条目(entries)都是缩进的。
%
% \DescribeMacro{\@dottedtocline}
% 大多数 |\l@kind| 命令是根据 |\@dottedtocline| 命令来定义的。此命令有三个参数:\\
% |\@dottedtocline{|\meta{seclevel}|}{|\meta{indent}|}{|\meta{numwidth}|}|. \\
% 例如,|\l@section| 命令的一个定义是:\\
% |\newcommand*{\l@section}{\@dottedtocline{1}{1.5em}{2.3em}}| \\
% 有必要更改条目的默认排版,那么通常有必要更改这些定义(但 \Lpack{tocloft}\ 包提供
% 了一些句柄[handles],可以轻松地更改内容,而无需知道 \LaTeX{}\ 内部组件)。
%
% 您可以使用 |\addcontentsline| 命令向文件中添加 |\contentsline| 命令。
%
% \DescribeMacro{\addtocontents}
% \LaTeX{}\ 还提供了 |\addtocontents{|\meta{file}|}{|\meta{text}|}| 命令,该命令将
% 把 \meta{text}\ 插入 \meta{file}。您可以使用它向文件中添加额外的(extra)文本和/或宏,
% 以便在文件由 |\tableofcontents| 命令(或任何其他用于 \meta{file}\ 处理的命令,
% 例如处理 \file{.lot}\ 文件的 |\listoftables| 命令)排版时进行处理。
%
% 当 |\addcontentsline| 和 |\addtocontents| 将其参数写入文件时,参数中使用的
% 任何脆弱命令(fragile commands)都必须受到 |\protect|ed。
%
% 您可以在不使用任何宏包的情况下对 ToC 等布局(layout)进行某些调整。例如:
% \begin{itemize}
% \item 如果您的页码(page numbers)突出到右边空白处(righthand margin),则可以进行如下设置:
%  \begin{verbatim}
%  \renewcommand{\@pnumwidth}{3em}  %设置“页码宽度”为3em
%  \renewcommand{\@tocrmarg}{4em}  %设置“标题文本的右边距缩进量”为4em
%  \end{verbatim}
%  \vspace{-2em}
% 以使长度(lengths)适合您文档。
%
% \item 为了在目录(ToC)等中有(节[sectional])标题(titles),右对齐并不加连字符(hyphenation)排版,则可以:
% \begin{verbatim}
% \renewcommand{\@tocrmarg}{2.55em plus1fil}
% \end{verbatim}
%   \vspace{-2em}
% 这里的值 |2.55em| 可以根据您需要的边距空间(margin space)进行更改。
%
% \item 引线中的点(dots in the leaders)可以通过将 |\@dotsep| 增加到一个大值来消除:
%  \begin{verbatim}
%  \renewcommand{\@dotsep}{10000}
%  \end{verbatim}
%
% \item 在您的目录(ToC)和图形目录(LoF)中如果想要有点引线(dotted leaders),但在表格目录(LoT)中不要有点引线,则可以:
% \vspace{-1em}
% \begin{verbatim}
% ...
% \tableofcontents
% \makeatletter \renewcommand{\@dotsep}{10000} \makeatother
% \listoftables
% \makeatletter \renewcommand{\@dotsep}{4.5} \makeatother
% \listoffigures
% ...
% \end{verbatim}
%  \vspace{-2em}
% 对于此文档,我使用此方法将图形目录(LoF)的点间距(dot spacing)与目录(ToC)的点间距加倍。
% 如您所见,所有点引线(dot leaders)具有相同的间距(spacing)要好得多。
%
% \item 要在部分条目(entry for a Part)的下面添加贯穿目录(ToC)整个宽度的水平线(horizontal line),请执行以下操作:
% \vspace{-0.5em}
% \begin{verbatim}
% \part{Part title}
% \addtocontents{toc}{\protect\mbox{}\protect\hrulefill\par}
% \end{verbatim}
%  \vspace{-2em}
% 请注意,当 |\addtocontents| 和 |\addcontentsline| 都将它们的参数写入文件时,
% 这意味着它们参数中的任何{\kaiti 脆弱}命令(\emph{fragile} commands)都必须通过在
% 每个脆弱命令前面加上 |\protect| 来保护。上述示例的结果将是 \file{.toc}\ 文件中的
% 以下两行(假设它是第二部分,位于第 34 页):
% \vspace{-0.5em}
% \begin{verbatim}
% \contentsline {part}{II\hspace {1em}Part title}{34}
% \mbox {}\hrulefill \par
% \end{verbatim}
% \vspace{-2em}
% 如果未使用 |\protect|s,则第二行将改为:
% \vspace{-0.5em}
% \begin{small}
% \begin{verbatim}
% \unhbox \voidb@x \hbox {}\unhbox \voidb@x \leaders \hrule \hfill \kern \z@ \par
% \end{verbatim}
% \end{small}
% \vspace{-2em}
% 您可能会在目录(ToC)中得到不希望的分页符(page breaks)。例如,您可能有一个很长的
% 多行节标题(long multiline section title),并且在 ToC 中,行与行之间有一个分页符。
% 在文档稳定(stable)之后,您可以在文档正文(body)的适当位置使用 |\addtocontents| 来
% 调整 ToC 中的分页。例如:
%  \begin{itemize}
%  \item |\addtocontents{toc}{\protect\newpage}| 强制分页。
%  \item |\addtocontents{toc}{\protect\enlargethispage{2\baselineskip}}| 使页面变长。
%  \item |\addtocontents{toc}{\protect\needspace{2\baselineskip}}| 指定如果页面上
%  没有两条基线(baselines)的垂直空间(vertical space),则开始一个新
%  页面(\Lpack{needspace}\ 宏包中定义了 |\needspace| 宏)。
%  \end{itemize}
%
% \end{itemize}
% 请记住,如果要修改任何包含 |@| 符号(sign)的命令,则必须在 \file{.sty}\ 文件中执行,
% 或者如果在文档本身中,必须用 |\makeatletter| 和 |\makeatother| 包围。例如,
% 如果你想在文档的前言(preamble)中修改 |\@dotsep|,您必须这样做:
% \begin{verbatim}
% \makeatletter
% \renewcommand{\@dotsep}{9.0}
% \makeatother
% \end{verbatim}
%
% \clearpage
% \section{\Lpack{tocloft}\ {\heiti 宏包}} \label{sec:usc}
%
% \Lpack{tocloft}\ 包提供了指定目录(Table of Contents,ToC)、图形目录
% (List of Figures,LoF)和表格目录(List of Tables,LoT)的排版方法。
%
% \DescribeMacro{\tableofcontents}
% \DescribeMacro{\listoffigures}
% \DescribeMacro{\listoftables}
% ToC、LoF 和 LoT 按照正常的(normal) \LaTeX\ 打印在文档中调用这些命令的位置。然而,
% 标准的(standard) \LaTeX{}\ 行为与 \Lpack{tocloft}\ 宏包的行为有一个区别。
% 在具有 |\chapter| 标题(headings)的标准 \LaTeX{}\ 类中,ToC、LoF 和 LoT 都显示在新页面上。
% 使用 \Lpack{tocloft}\ 宏包,它们不必开始新的页面;如果您想让它们出现在新页面上,
% 您必须事先发出适当的命令,例如:
% \vspace{-0.5em}
% \begin{verbatim}
%  ...
% \clearpage
% \tableofcontents
% \clearpage
% \listoftables
% ...
% \end{verbatim}
% \vspace{-1em}
% ToC、LoF 和/或 LoT 的 |\thispagestyle| 页面样式(page style)\\由 |\tocloftpagestyle{|\meta{style}|}| 命令
% \DescribeMacro{\tocloftpagestyle}
% 设置,其中 \meta{style}\ 是可用的页面样式之一。该宏包最初设置为 |\tocloftpagestyle{plain}|。
%
% \subsection[宏包选项]{\heiti 宏包选项}
%
% 该宏包具有以下选项:
% \begin{itemize}
% \item[\Lopt{subfigure}] 只有当 \Lpack{tocloft}\ 和 \Lpack{subfigure}\ 包一起使用时,
% 才需要此选项。这两个包可以按任何顺序指定。
%
% \item[\Lopt{titles}] \Lopt{titles}\ 选项使用默认的 \LaTeX{}\ 方法对 ToC、 LoF 和 LoT 的标题(titles)进行排版。
% 这可能是有用的,例如,当 \Lpack{tocloft}\ 和 \Lpack{fncychap}\ 宏包一起使用时,
% “花哨的(fancy)”章样式(chapter styles)应该用于 ToC 等标题。
%
% \end{itemize}
%
% 如果您使用 \Lopt{titles}\ 选项,可以忽略下一节,并继续阅读第~\ref{sec:entries}~节。
%
% \subsection[更改标题]{\heiti 更改标题} \label{sec:titles}
%
% 提供了用于控制标题外观(appearance of the titles)的命令。按照 \LaTeX{}\ 自定义(custom),
% 标题文本(title texts)是 |\contentsname|、|\listfigurename| 和 |\listtablename| 命令的值。
%
% 为 ToC、LoF 和 LoT 标题排版控件(title typsetting control)提供了类似的命令集(sets of commands)。
% 为了方便(当然是我的,也希望是你的),在下面的描述中,我将使用 |Z| 来代表“toc”或“lof”或“lot”。
% 例如,|\cftmarkZ| 代表 |\cftmarktoc| 或 |\cftmarklof| 或 |\cftmarklot|。
%
% \DescribeMacro{\cftmarkZ}
% 这些宏(macros)设置了 ToC、LoF 和 LoT 页面上的短标题(running heads)。你可能不需要改变这些。
%
% \DescribeMacro{\cftbeforeZtitleskip}
% \DescribeMacro{\cftafterZtitleskip}
% 这些长度控制(lengths control)标题前后的垂直间距(vertical spacing)。
% 您可以使用 |\setlength| 更改它们的默认值。
%
% \DescribeMacro{\cftZtitlefont}
% \DescribeMacro{\cftafterZtitle}
% 用于排版 ToC 标题的代码如下所示:
% \begin{verbatim}
% {\cfttoctitlefont \contentsname}{\cftaftertoctitle}\par
% \end{verbatim}
% \vspace{-1em}
% 默认情况下,|\cftZtitlefont| 被定义为字体规范(font specification)(例如,|\Large\bfseries|),
% 并且 |\cftafterZtitle| 为空。可以(通过 |\renewcommand|)更改这些命令进而更改排版,例如:
% \begin{itemize}
% \item |\renewcommand{\cftZtitlefont}{\hfill\Large\itshape}| 将排版出大斜体(Large italic)右对齐(flushright)的标题(title)。
% \item |\renewcommand{\cftZtitlefont}{\hfill\Large\bfseries}| 与
%
% |\renewcommand{\cftafterZtitle}{\hfill}| 将
% 排版出居中的大粗体标题(Large bold title)。
% \item 按下面这样做:
% \vspace{-0.5em}
% \begin{verbatim}
% \renewcommand{\cftafterZtitle}{%
%   \\[\baselineskip]\mbox{}\hfill{\normalfont Page}}
% \end{verbatim}
% \vspace{-2em}
%       将在标题(title)后面的一行中加上“Page”一词。(如果这样做,则可能需要减少 |\cftafterZtitleskip|)。
% \item |\renewcommand{\cftafterZtitle}{\thispagestyle{empty}}| 将使页面标题为空(即,不会打印页码)。
% \end{itemize}
%
%
% \subsection[排版条目]{\heiti 排版条目} \label{sec:entries}
%
% 还提供了命令,以便能够更好地控制不同类型条目的排版。定义条目的默认布局(default layout)的参数作
% 为 \Lpack{layouts}\ 宏包的一部分或在~\cite[page 34]{GOOSSENS94}~中进行了说明,
% 并在图~\ref{fig:ltoc}~中重复。
%
% \DescribeMacro{\Zdepth}
% |\Zdepth{|\meta{number}|}| 命令与标准的 |\tocdepth{|\meta{number}|}| 命令类似,
% 因为它规定,如果新列表(new listing)中的条目的编号级别(numbering level)大于 \meta{number},
% 则不应该对它们进行排版。默认定义为 |\setcounter{Zdepth}{1}|。这些命令是必需的,
% 例如 |subcaption| 的用户需要这些命令,|subcaption| 将生成对应于 |lofdepth| 和 |lotdepth| 为 2 的
% 子图标题(subfigure captions)和子表标题(subtable captions)。
%
% 在默认的 ToC 排版中,只有较小的条目(minor entries)在节标题(sectioning title)和
% 页码(page number)之间有点引线(dotted leader lines)。\Lpack{tocloft}\ 宏包为
% 所有条目提供一般引线(general leaders)。
% \DescribeMacro{\cftdot}
% 引线中的“点(dot)”由 |\cftdot| 的值给出。
% 它的默认定义是 |\newcommand{\cftdot}{.}|,它给出了默认的点引线。通过更改 |\cftdot| 您
% 可以在引线中使用句号(period)以外的符号(symbols),例如:
% \begin{verbatim}
% \renewcommand{\cftdot}{\ensuremath{\ast}}
% \end{verbatim}
% \vspace{-2em}
% 将会生成一个以星号(asterisks)作为符号(symbol)的引线。
%
% 每种条目都可以控制其前导(见下文)中的点(dots)之间的间隔。但是,为了保持一致性,
% 所有点引线(dotted leaders)应使用相同的间距(same spacing)。宏 |\cftdotsep| 指定默认间距(default spacing)。
% 它的值是一个数字。
% \DescribeMacro{\cftdotsep}
% 但是,如果间隔(separation)太大,则实际上没有点(dots)会被排版。
% \DescribeMacro{\cftnodots}
% 宏 |\cftnodots| 是一个“很大的(too large)”的分隔值(separation value)。
%
% \DescribeMacro{\cftsetpnumwidth}
% 页码(page numbers)在固定宽度的盒子中排版。|\cftsetpnumwidth{|\meta{length}|}| 命令可用于
% 更改盒子的宽度(\LaTeX\ 的内部 |\@pnumwidth|)。标题文本(title texts)将在到达右边距(righthand margin)之前结束。
% \DescribeMacro{\cftsetrmarg}
% |\cftsetrmarg{|\meta{length}|}| 可用于设置该距离(\LaTeX\ 的内部 |\@tocrmarg|)。请注意,
% |\cftsetrmarg| 中使用的长度(length)应大于 |\cftsetpnumwidth| 中设置的长度。这些值在任何给定文档中都应保持不变。
%
% 页码(page numbers)在如上所述的盒子中排版。默认情况下,它们是右对齐的(right-aligned),当页码
% 在页面上垂直对齐以便它们数字对齐(digits line up)时,右对齐是合适的。例如,对于 ToC 条目与其
% 页码之间具有固定宽度(fixed width)的设计,左对齐(left alignment)可能更合适。
% \DescribeMacro{\cftpnumalign}
% 这可以通过将 \cs{cftpnumalign}\ 宏设置为 \verb|l|、\verb|c| 或 \verb|r|(就像 \cs{makebox})来控制:
% \begin{verbatim}
%   \renewcommand{\cftpnumalign}{l}
% \end{verbatim}
% \vspace{-1.5em}
%
% \DescribeMacro{\cftparskip}
% 通常,ToC 等中的 |\parskip| 为零。这可以通过更改 |\cftparskip| 长度(length)进行更改。
% 请注意,|\cftparskip| 的当前值(current value)用于目录(ToC)、图形目录(LoF)和表格目录(LoT),
% 但如果其中一个或另一个具有不同的值(这不是一个好主意),则可以在调
% 用 |\tableofcontents| 或 |\listoffigures| 或 |\listoftables| 之前更改该值。
%
%
%    我们使用 |X| 代表以下内容:
% \begin{itemize}
% \item 用 |part| 代表 |\part| 标题(titles)
% \item 用 |chap| 代表 |\chapter| 标题(titles)
% \item 用 |sec| 代表 |\section| 标题(titles)
% \item 用 |subsec| 代表 |\subsection| 标题(titles)
% \item 用 |subsubsec| 代表 |\subsubsection| 标题(titles)
% \item 用 |para| 代表 |\paragraph| 标题(titles)
% \item 用 |subpara| 代表 |\subparagraph| 标题(titles)
% \item 用 |fig| 代表图形(figure) |\caption| 标题(titles)
% \item 用 |subfig| 代表子图形(subfigure) |\caption| 标题(titles)
% \item 用 |tab| 代表表格(table) |\caption| 标题(titles)
% \item 用 |subtab| 代表子表格(subtable) |\caption| 标题(titles)
% \end{itemize}
%
% \DescribeMacro{\cftbeforeXskip}
% 这将控制条目前的垂直空间(vertical space)。可以使用 |\setlength| 来更改。
%
% \DescribeMacro{\cftXindent}
% 这控制从左边距(left margin)缩进条目(图~\ref{fig:ltoc}~中的{\kaiti 缩进}[\textit{indent}])。
% 可以使用 |\setlength| 来更改。
%
% \DescribeMacro{\cftXnumwidth}
% 这控制了排版标题编号(title numbers)所允许的空间(图~\ref{fig:ltoc}~中的{\kaiti 编号宽度}
% [\textit{numwidth}])。可以使用 |\setlength| 来更改。多行标题(multiline title)的第二行
% 和后续行(subsequent lines)将缩进此量。
%
% 其余的命令与条目排版的细节(specifics)有关。这是一个简化的伪代码版本(pseudo-code version),
% 用于已编号条目(numbered entries)和未编号条目(unnumbered entries)的排版。
% \begin{verbatim}
% {\cftXfont {\cftXpresnum SNUM\cftXaftersnum\hfil} \cftXaftersnumb TITLE}%
%         {\cftXleader}{\cftXpagefont PAGE}\cftXafterpnum\par
%
% {\cftXfont TITLE}{\cftXleader}{\cftXpagefont PAGE}\cftXafterpnum\par
% \end{verbatim}
% 其中 |SNUM| 是节编号(section number),|TITLE| 是标题文本(title text),|PAGE| 是页码(page number)。
% 在已编号的条目(numbered entry)中,伪代码(pseudo-code)\\
% |{\cftXpresnum SNUM\cftaftersnum\hfil}| \\
% 是在宽度为 |\cftXnumwidth| 的盒子内排版的。
%
% \DescribeMacro{\cftXfont}
% 这将控制标题(及其前面的编号,如果有)的外观。可以使用 |\renewcommand| 命令来更改。
%
% \changes{v1.1}{2000/02/11}{Added description of \cs{cftXpresnum}}
% 通常,节编号(section number)是在宽度为 |\cftXnumwidth| 的盒子内排版的。在盒子中,
% \DescribeMacro{\cftXpresnum}
% 首先调用宏 |\cftXpresnum|,然后是排版编号(number),然后是在排版编号之后调用 |\cftXaftersnum| 宏。
% 盒子中的最后一个命令是 |\hfil|,
% \DescribeMacro{\cftXaftersnum}
% 它用于使盒子中的内容左对齐(flushleft)。盒子排版后,
% 在排版标题文本(title text)之前调用 |\cftXaftersnumb| 宏。
% \DescribeMacro{\cftXaftersnumb}
% 所有这三个都可以通过 |\renewcommand| 进行更改。
% 默认情况下,它们被定义为什么都不做。
%
% 在标准类(standard classes)中,|\part| 的 ToC 条目只是排版为编号(number)和标题(title),
% 后跟页码(page number),在排版编号和标题之前调用 |\cftpartpresnum| 宏。由于 \LaTeX\ ideosyncracies,
% 如果第三方宏包的行为与默认的内部 \LaTeX\ 命令不同,|\cftpartpresnum| 在输出中可能会加倍(doubled)。
% |tocloft| 宏包包含特定代码,以防止 KomaScript 类和 |titlesec| 宏包出现这种情况;
% 如果您发现其他宏包也表现出这种错误行为(mis-behaviour),请联系维护人员添加进一步的更正。
%
% 当使用标准类(standard class)时,|\cftpartaftersnum| 和 |\cftpartaftersnumb| 宏无效,
% 但如果使用非标准类(non-standard class),它们可能会起作用。
%
%
% \DescribeMacro{\cftXleader}
% |\cftXleader| 定义标题(title)和页码(page number)之间的引线(leader),
% 它可以通过 |\renewcommand| 进行更改。引线中任何点(dots)之间的间距(spacing)
% 由 |\cftXdotsep| (图~\ref{fig:ltoc}~中的 |\@dotsep|)控制。
% \DescribeMacro{\cftXdotsep}
% 它可以通过 |\renewcommand| 进行更改,其值必须是数字
% (例如 6.6 或 |\cftdotsep|)或 |\cftnodots|(禁用点[dots])。间距以{\kaiti 数学单位}(\emph{math units})表示,
% 例如“18mu to 1em”。
%
% \DescribeMacro{\cftXpagefont}
% 这定义了用于排版页码(page number)的字体。它可以通过 |\renewcommand| 进行更改。
%
% \DescribeMacro{\cftXafterpnum}
% 此宏在页码排版后调用。它的默认值是什么都不做。它可以通过 |\renewcommand| 进行更改。
%
% \DescribeMacro{\cftsetindents}
% 命令 |\cftsetindents{|\meta{entry}|}{|\meta{indent}|}{|\meta{numwidth}|}| 设
% 置 \meta{entry}\ 的{\kaiti 缩进}(\textit{indent})为长度 \meta{indent},设置 \meta{entry}\
% 的{\kaiti 编号宽度}(\textit{numwidth})为长度 \meta{numwidth}。\meta{entry}\ 参数是
% 一个标准条目(standard entries)(例如,subsection [子节])的名称或用 \Lpack{tocloft}\ 宏包定义的条目的名称。
% 例如,|\cftsetindents{figure}{0em}{1.5em}| \\ 将使图形(figure)条目左对齐(left justified)。
%
% 通过单独或组合更改 |\cftXfont|、|\cftXaftersnum|、|\cftXaftersnumb|、|\cftXleader| 和
% |\cftXafterpnum| 的定义,可以实现各种效果(effects)。为了示例起见,假设我们有以下初始定义(initial definitions):
% \begin{verbatim}
% \newcommand{\cftXfont}{}
% \newcommand{\cftXaftersnum}{}
% \newcommand{\cftXaftersnumb}{}
% \newcommand{\cftXleader}{\cftdotfill{\cftXdotsep}}
% \newcommand{\cftXdotsep}{\cftdotsep}
% \newcommand{\cftXpagefont}{}
% \newcommand{\cftXafterpnum}{}
% \end{verbatim}
% \vspace{-1.5em}
% (请注意,标题[title]、页眉[leader]和页码[page number]应使用相同的字体,以提供一致的外观[coherent appearance])。
%
% \begin{itemize}
% \item 要消除引线中的点(dots in the leader),请执行以下操作:
% \vspace{-1.0em}
% \begin{verbatim}
% \renewcommand{\cftXdotsep}{\cftnodots}
% \end{verbatim}
% \vspace{-2.0em}
% \item 把某物(如名字)放在标题(编号)之前,请执行以下操作:
% \vspace{-1.0em}
% \begin{verbatim}
% \renewcommand{\cftXpresnum}{SOMETHING }
% \end{verbatim}
% \vspace{-2.0em}
% \item 要在节编号(section number)后添加冒号(colon),请执行以下操作:
%% \vspace{-1.0em}
% \begin{verbatim}
% \renewcommand{\cftXaftersnum}{:}
% \end{verbatim}
% \vspace{-2.0em}
% \item 要在标题编号(title number)之前添加内容,在标题编号之后添加冒号(colon),将所有内容设置为
% 粗体字体(bold font),并在下一行开始标题文本(title text),请执行以下操作:
% \vspace{-1.0em}
% \begin{verbatim}
% \renewcommand{\cftXfont}{\bfseries}
% \renewcommand{\cftXleader}{\bfseries\cftdotfill{\cftXdotsep}}
% \renewcommand{\cftXpagefont}{\bfseries}
% \renewcommand{\cftXpresnum}{SOMETHING }
% \renewcommand{\cftXaftersnum}{:}
% \renewcommand{\cftXaftersnumb}{\\}
% \end{verbatim}
%
% 如果在编号盒子(number box)中除了编号(number)之外还要添加文本(text),则可能需要增加盒子的宽度,
% 以便多行标题(multiline titles)具有整齐的垂直对齐方式(neat vertical alignment);
% 更改盒子宽度通常意味着缩进(indents)也需要修
% 改\ \footnote{Lyndon Dudding(林登·杜丁)(\texttt{lyndon.dudding@totalise.co.uk})发现了这一点。}。
% 调整上述示例的盒子宽度(box width)的一种可能方法是:
% \vspace{-1.0em}
% \begin{verbatim}
% \newlength{\mylen}   % a "scratch" length
% \settowidth{\mylen}{\bfseries\cftXpresnum\cftXaftersnum} % 额外空格
% \addtolength{\cftXnumwidth}{\mylen} % 添加额外空格(extra space)
% \end{verbatim}
% \vspace{-2.0em}
% \item 要将节编号(section numbers)设置为右对齐(flushright)\footnote{感谢 David Holz(大卫·霍尔茨)
% (\texttt{lbda@earthlink.net})的请求。},执行:
% \vspace{-1.0em}
% \begin{verbatim}
% \setlength{\mylen}{0.5em}   % 在编号末尾需要一些额外的空格
% \renewcommand{\cftXpresnum}{\hfill}  % 注意两个“l”
% \renewcommand{\cftXaftersnum}{\hspace*{\mylen}}
% \addtolength{\cftXnumwidth}{\mylen}
% \end{verbatim}
% \vspace{-2.0em}
% 在上文中,盒子中添加的初始(initial) |\hfill| 将覆盖盒子中的最终(final) |\hfil|,
% 从而将所有内容移动到盒子的右端。额外的空格(extra space)是为了使数字(number)不会
% 直接排版在标题文本的左侧(left of the title text)。
%
% \item 要将条目设置为左对齐(ragged left)(但这只适用于单行标题[single line titles]):
% \vspace{-1.0em}
% \begin{verbatim}
% \renewcommand{\cftXfont}{\hfill\bfseries}
% \renewcommand{\cftXleader}{}
% \end{verbatim}
% \vspace{-2.0em}
% \item 要将页码(page number)设置在条目文本(entry text)之后而不是右侧边距(righthand margin):
% \vspace{-1.0em}
% \begin{verbatim}
% \renewcommand{\cftXleader}{}
% \renewcommand{\cftXafterpnum}{\cftparfillskip}
% \renewcommand{\cftpnumalign}{l}
% \end{verbatim}
% \vspace{-2.0em}
% 默认情况下,|\parfillskip| 值在局部设置为填充(fill up)段落的最后一行。只更改 |\cftXleader| 会
% 在标题的最后一行放置讨厌的字间空格(interword spaces)。|\cftparfillskip| 命令
% 是 \Lpack{tocloft}\ 宏包的一部分,提供该命令只是为了实现上述效果。此外,这是一个
% 很好的例子,说明何时适合更改页码盒子(page number box)的对齐方式。
%
% \item 要删除表格(table)和图形(figure)标题条目(caption entries)之间插入的空格,请在章(chapters)之间:
% \vspace{-1.0em}
% \begin{verbatim}
% \begingroup
%   \renewcommand*{\addvspace}[1]{}
%   \listoftables
%   \listoffigures
% \endgroup
% \end{verbatim}
% \end{itemize}
% \vspace{-2.0em}
% \DescribeMacro{\cftpagenumbersoff}
% 命令 |\cftpagenumbersoff{|\meta{entry}|}| 将删除列表(listing)中 \meta{entry}\ 的
% 页码(page numbers),其中 \meta{entry}\ 是标准类型的条目(standard kinds of entries)之
% 一的名称(例如,|subsection| 或 |figure| ---  如果使用了 \Lpack{subfigure}\ 则包
% 括 |subfigure| --- 等等),或由 \Lpack{tocloft}\ 宏包定义的新条目的名称。
%
% \DescribeMacro{\cftpagenumberson}
% 命令 |\cftpagenumberson{|\meta{entry}|}| 反转(reverses)相应的 |\cftpagenumbersoff| 的效果。
%
% 在 \file{comp.text.tex}\ 新闻组(newsgroup)中出现的一个问题是,如何在{\kaiti 没有}
% 页码的情况下获取目录(ToC)中的附录清单(Appendices list)的标题。这里有一个简单的方法,
% 假设文档包含章(chapters)
% \vspace{-1.0em}
% \begin{verbatim}
% ...
% \appendix
% \addtocontents{toc}{\cftpagenumbersoff{chapter}}
% \chapter{First appendix}
% \end{verbatim}
% \vspace{-2.0em}
% 如果在附录(appendices)之后还有其他章标题(chaptered headings)要进入目录(ToC),
% 那么就有必要执行 \\
% |\addtocontents{toc}{\cftpagenumberson{chapter}}| \\
% 这样类似的工作以恢复目录(ToC)中的页码(page numbering)。
%
% 类似地,如果您使用的是 \Lpack{subfigure}\ 宏包,您可能需要删除子图片标题(subfigure captions)的页码,
% 这可以通过以下方式完成:\\
% |\cftpagenumbersoff{subfigure}|
%
% 在这一点上,我把它留给你的独创性(ingenuity),使你可以实现其他效果。但是,如果您
% 想到了更多的示例,请让我知道是否可能包含在本文档的后续版本中。
%
% \subsection[新列表\ \ldots]{{\heiti 新列表}\ \ldots}
%
% \DescribeMacro{\newlistof}
% {\large \color{blue}{|\newlistof[|\meta{within}|]{|\meta{entry}|}{|\meta{ext}|}{|\meta{listofname}|}|}}\\[4pt]
% 命令创建一个新列表\ \ldots\ (new List of \ldots),以及与之配套的各种命令。
%
% 第一个必需的参数 \meta{entry}\ 用于定义一个名为 |entry| 的新计数器(counter)。
% 可以使用可选的 \meta{within}\ 参数,每次更改 |within| 调用的计数器(counter)时,
% |entry| 都重置为 1。也就是说,前两个参数等同于调用|\newcounter{|\meta{entry}|}[|\meta{within}|]|。
%
% 下一个参数 \meta{ext}\ 是新列表(new List)的文件扩展名(file extension)。最后一个
% 参数 \meta{listofname}\ 是新列表(new List)的标题文本(text for the heading)。举个例子:
% \begin{verbatim}
% \newcommand{\listanswername}{List of Answers}
% \newlistof[chapter]{answer}{ans}{\listanswername}
% \end{verbatim}
% \vspace{-2.0em}
% 将创建一个新的 |answer| 计数器,该计数器将在每一 |\chapter{...}| 开始时被重置(reset)。
% 任何答案标题(answer titles)都将写入文件 \file{jobname.ans},并且 |\listanswername| 将
% 用作列表标题(list heading)。创建的命令 |\listofanswer| 可以
% 像 |\listoftables| 或 |\tableofcontents| 命令一样用来生成一个列表(listing)。
% 由您来指定如何将这些条目放入新的答案列表(new List of Answers)中。
% 下面是一个非常简单的示例,请记住已经创建了一个 |answer| 计数器。
% \begin{verbatim}
% \newcommand{\answer}[1]{%
%   \refstepcounter{answer}
%   \par\noindent\textbf{Answer \theanswer. #1}
%   \addcontentsline{ans}{answer}{\protect\numberline{\theanswer}#1}\par}
% \end{verbatim}
% \vspace{-1.5em}
% 当使用时,如:\\
% |\answer{Hard} The \ldots|\  将打印为:
% \par\noindent\textbf{Answer 1. Hard}\par The \ldots
%
% \vspace{2em}
% 如上所述,|\newlistof| 命令创建了几个新命令,其中大部分您现在应该熟悉了。为方便起见,
% 假设 |\newlistof{X}{Z}{...}| 已发布,因此,|X| 是新计数器的名称,与
% 第~\ref{sec:entries}~节中的 |X| 相对应,|Z| 是新文件扩展名,与
% 第~\ref{sec:titles}~节中的 |Z| 相对应。然后,除其他外,将提供以下新命令。
%
% 这五个命令 |\cftmarkZ|、|\cftbeforeZtitleskip|、|\cftafterZtitleskip|、|\cftZtitlefont| 和 |\cftafterZtitle|,
% 与第~\ref{sec:titles}~节中描述的同名命令类似。
% 命令 |\listofX| 类似于 |\listoftables| 等,
% \DescribeMacro{\listofX}
% 因为它在调用新列表(new listing)的位置排版新列表。
%
% \DescribeMacro{\Zdepth}
% |\Zdepth{|\meta{number}|}| 命令类似于标准的 |\tocdepth{|\meta{number}|}| 命令,
% 因为它指定,如果新清单(new listing)中的条目(entries)的编号级别(numbering level)大
% 于 \meta{number},则不应该对它们进行排版。默认定义为 |\setcounter{Zdepth}{1}|。
%
% \vspace{1em}
% \DescribeMacro{\newlistentry}
% {\large \color{blue}{|\newlistentry[|\meta{within}|]{|\meta{entry}|}{|\meta{ext}|}{|\meta{level-1}|}|}}\\[4pt]
% 命令创建用于在列表(listing)中排版新条目的新命令。它由 |\newlistof| 命令内部使用,但可以独立使用。
%
% 第一个必需参数 \meta{entry}\ 用于定义名为 |entry| 的新计数器。
% 可以使用可选的 \meta{within}\ 参数,每次更改 |within| 调用的计数器(counter)时,
% |entry| 都重置为 1。也就是说,前两个参数等同于调用 |\newcounter{|\meta{entry}|}[|\meta{within}|]|。
% 第二个必需参数 \meta{ext},是条目列表(entry listing)的文件扩展名(file extension)。
% 最后一个参数 \meta{level-1}\ 是一个数字,指定清单中条目的编号级别(numbering level)减去 1。
% 例如,|\newlistof[chapter]{answer}{ans}{\listanswername}| 命令将调用 |\newlistentry[chapter]{answer}{ans}{0}| 命令。
%
% 调用 |\newlistentry| 会创建几个新命令。假设它被称为\\ |\newlistentry[within]{X}{Z}{N}|,
% 其中 |X| 和 |Z| 与前面的用法类似,并且 |N| 是一个整数,那么以下命令可用。
%
% |\cftbeforeXskip|、|\cftXfont|、|\cftXpresnum|、|\cftXaftersnum|、\\
% |\cftXaftersnumb|、|\cftXleader|、|\cftXdotsep|、|\cftXpagefont| 和 |\cftXafterpnum|,
% 这组命令与第~\ref{sec:entries}~节中描述的同名命令类似。它们的默认值也如前所述。
%
% |\cftXindent| 和 |\cftXnumwidth| 的默认值是根据 \meta{level-1}\ 参数(例如,在本例中是 |N|)的值设置的。
% 对于 |N=0|,设置对应于非分章文档(non-chaptered documents)中节(sections)的设置,如表~\ref{tab:indents}~所示。
% 对于 |N=4|,设置对应于非分章文档(non-chaptered documents)中的子段(subparagraphs)的设置。
% 而对于 |0<N<4| 的中间值(intermediate values),对应于分章文档(chaptered documents)中匹配的节划分(sectional division)。
% 对于 |N<0| 或 |N>4| 的值,或对于非默认值(non-default values),使用 |\cftsetindents| 命令设置值。
%
%
% \DescribeMacro{\l@X}
% |\l@X| 是一个内部命令(internal command),用于在排版列表(listing)中的条目,并根据上述 |\cft*X*| 命令
% 进行定义。如果 |\Zdepth| 为 |N| 或更小,则不会排版条目,其中 |Z| 是列表的文件扩展名。
%
% \DescribeMacro{\theX}
% 命令 |\theX| 打印计数器 |X| 的值。它最初被定义为打印阿拉伯数字(arabic numerals)。
% 如果使用可选的 \meta{within}\ 参数,\\则 |\theX| 定义为:|\renewcommand{\theX}{\thewithin.\arabic{X}}|,\\
% 否则,|\theX| 定义为:|\renewcommand{\theX}{\arabic{X}}|.
%
%
% 作为独立使用 |\newlistentry| 的示例,下面将设置子答案(sub-answers)。
% \begin{verbatim}
% \newlistentry[answer]{subanswer}{1}
% \cftsetindents{subanswer}{1.5em}{3.0em}
% \renewcommand{\thesubanswer}{\theanswer.\alph{subanswer}}
% \newcommand{\subanswer}[1]{%
%    \refstepcounter{subanswer}
%    \par\textbf{\thesubanswer) #1}
%    \addcontentsline{ans}{subanswer{\protect\numberline{\thesubanswer}#1}}
% \setcounter{ansdepth}{2}
% \end{verbatim}
% \vspace{-2em}
% 然后:
% \vspace{-0.5em}
% \begin{verbatim}
% \answer{Harder} The \ldots
%   \subanswer{Reformulate the problem} 它有助于 \ldots
% \end{verbatim}
% \vspace{-1.5em}
% 将排版为:
% \par\noindent\textbf{Answer 2. Harder}\par The \ldots
% \par\textbf{2.a) Reformulate the problem} 它有助于 \ldots
%
% \vspace{2em}
% 默认情况下,答案条目(answer entries)将出现在答案列表(List of Answers listing)中
% (由 |\listofanswer| 命令排版)。为了显示子答案(subanswers),
% 上面使用了 |\setcounter{ansdepth}{2}| 命令。
%
% 若要关闭子答案(subanswers)的页码,请执行\\
% |\cftpagenumbersoff{subanswer}|
%
% 作为 |\newlistentry| 的另一个例子,假设在 |subparagraph| 之下需要额外的
% 节划分(sectioning division),称为子段(|subsubpara|)。|\subsubpara| 命令本身可以
% 通过 LaTeX 内核(kernel)命令 |\@startsection| 来定义。此外,还需要定义一个 |\subsubparamark| 宏、
% 一个新的 |subsubpara| 计数器、一个 |\thesubsubpara| 宏和一个 |\l@subsubpara| 宏。
% 如下所示,使用 \Lpack{tocloft}\ 宏包的 |\newlistentry| 可以解决如下所示的大部分问题
% (请记住关于带有 |@| 符号的命令的注意事项)。
% \begin{verbatim}
% \newcommand{\subsubpara}{\@startsection{subpara} %
%    {6}                             % 级别(level)
%    {\parindent}                    % 从左边距(left margin)缩进
%    {3.25ex \@plus1ex \@minus .2ex} % 标题上方间距(skip above heading)
%    {-1em}               % 页眉栏标题(runin heading)与标题和文本间有 1em
%    {\normalfont\normalsize\itshape} % 斜体数字和标题
% }
% \newlistentry[subparagraph]{subsubpara}{toc}{5}
% \cftsetindents{subsubpara}{14.0em}{7.0em}
% \newcommand*{\subsubparamark}[1]{}  % 吞噬标题标记(heading mark)
% \end{verbatim}
%
% \vspace{-1.5em}
% 每个 List off\ldots 使用一个文件来存储列表条目(list entries),这些文件在整个文档
% 处理(document processing)过程中必须保持打开状态以便写入。TeX 只有有限数量的文件
% 可以保持打开状态,这限制了可以使用的列表(listings)的数量。对于一个包含目录(ToC)
% 但没有其他额外辅助文件(ancillary files)(例如,没有索引或参考文献输出文件)的文档,
% 包括图形目录(LoF)和表格目录(LoT)在内的 LoX 的最大数量不能超过 11 个。如果您尝试
% 创建太多的新列表(new listings),LaTeX 将会返回错误消息:
% \begin{center}
% |No room for a new write| \ \ (译者注:没有空间以供新的写入)
% \end{center}
% 如果您收到这样的信息,唯一的办法就是重新设计(redesign)您的文档。
%
% \Lpack{tocloft}\ 宏包没有提供简单的方法用于指定新的浮动列表(Lists of Floats)或
% 浮动环境(float environments)。对于这些,我建议使用 \Lpack{ccaption}\ 宏包~\cite{CCAPTION}。
%
% \subsection[实验性程序]{\heiti 实验性程序}
%
% 本节中描述的宏比之前描述的宏更具实验性(experimental)。
%
% 一些老式小说,甚至一些现代教科书\ \footnote{例如,Robert Sedgewick (罗伯特·塞奇威克),
% 《{\kaiti 算法}(\textit{Algorithms})》,Addison-Wesley (艾迪生-韦斯利),1983 年。},
% 都在章标题(chapter heading)之后或在目录(Toc)中,或在两个地方,都有一个简短的章内容提要(synopsis)。
%
% \DescribeMacro{\cftchapterprecis}
% 命令 |\cftchapterprecis{|\meta{text}|}| 在文档中调用它的位置处打印它的参数,
% 并将它添加到 \file{.toc}\ 文件,例如:
% \begin{verbatim}
% ...
% \chapter{} % first chapter
% \cftchapterprecis{介绍了我们的英雄;家谱;早期。}
% ...
% \end{verbatim}
%
% \vspace{-1.5em}
% \DescribeMacro{\cftchapterprecishere}
% \DescribeMacro{\cftchapterprecistoc}
% |\cftchapterprecis| 命令调用这两个命令来打印文档(此处的 |\...here{|\meta{text}|}| 命令)
% 中的文本并将其放入目录(ToC)(此处的 |\...toc{|\meta{text}|}| 命令)。如果需要,可以单独使用。
%
% 有时可能需要更改单个条目的全局参数(global parameters)。例如,一幅图形(figure)可能被放在
% 一本书的最后一页(封面或封底的内侧),并需要放在图形目录(LoF)中,其页码设置为“inside front cover(封面内侧)”。
% 如果“inside front cover(封面内侧)”被排版为普通页码(ordinary page number),它将突出到页边空白处(margin)。
% 因此,需要更改此特定条目(particular entry)的参数。
%
% \DescribeMacro{\cftlocalchange}
% 命令 |\cftlocalchange{|\meta{file}|}{|\meta{pnumwidth}|}{|\meta{tocrmarg}|}| 将
% 在 \meta{file}\ 中写入一个条目(entry)以重置全局参数(global parameters)。
% 应该在任何特殊条目之后再次调用该命令,以将参数重置回其通常的值(usual values)。
% 必须保护参数中使用的任何脆弱命令(fragile commands)。
%
% \DescribeMacro{\cftaddtitleline}
% 命令 |\cftaddtitleline{|\meta{file}|}{|\meta{kind}|}{|\meta{title}|}{|\meta{page}|}| 将
% 在 \meta{file}\ 中写入一个 |\contentsline| 条目,作为一个 \meta{kind}\ 条目,
% 其标题为 \meta{title}\ 而页码为 \meta{page}。也就是说,条目的形式(form)是:\\
% |\contentsline{kind}{title}{page}| \\
% 必须保护参数中使用的任何脆弱命令(fragile commands)。
%
% \DescribeMacro{\cftaddnumtitleline}
% 命令 |\cftaddnumtitleline{|\meta{file}|}{|\meta{kind}|}{|\meta{num}|}{|\meta{title}|}{|\meta{page}|}| 类似,
% 只是它还包含 \meta{num}\ 作为 |\numberline| 的参数。也就是说,条目的形式(form)是:\\
% |\contentsline{kind}{\numberline{num} title}{page}| \\
% 必须保护参数中使用的任何脆弱命令(fragile commands)。
%
% 作为使用这些命令的示例,请注意,|\@pnumwidth| 和 |\@tocrmarg| 的默认 \LaTeX{}\ 值
% 分别为 1.55em 和 2.55em,您可以对扉页(frontispiece page)中的图形执行以下操作:
% \begin{verbatim}
% ...
% % 这是没有页码的扉页(frontispiece page)
% % 绘制或导入图形(没有 \caption)
% \cftlocalchange{lof}{4em}{5em} % 使扉页(frontispiece)的
%                   % 页码宽度(pnumwidth)足够大,并更改边距(margin)以适合
% \cftaddtitleline{lof}{figure}{The title}{frontispiece}
% \cftlocalchange{lof}{1.55em}{2.55em} % 恢复正常设置
% ...
% \end{verbatim}
% 回想一下,|\caption| 命令将在 \file{.lof}\ 文件中放入一个条目(entry),这在这里是不需要的。
% 如果需要标题(caption),您可以自己制作一个,或者,假设您的一般标题(general captions)不是太奇特(exotic),
% 则可以使用 \Lpack{ccaption}\ 宏包中的 |\legend| 命令。如果插图(illustration)已编号,
% 请使用 |\cftaddnumtitleline| 命令而不是 |\cftaddtitleline|。
%
% \DescribeMacro{\cftZprehook}
% \DescribeMacro{\cftZposthook}
% 实现多栏目录(multicolumn ToCs)出人意料地困难,您能猜到写下面这段话的问题是什么吗?
% \vspace{-0.5em}
% \begin{verbatim}
% \begin{multicols}{2}
% \tableofcontents
% \end{multicols}
% \end{verbatim}
% \vspace{-1.5em}
% 在普通(regular) \LaTeX{}\ 中做到这一点最简单的方法可能是:
% \vspace{-0.5em}
% \begin{verbatim}
% \RequirePackage{multicol}
% \AtBeginDocument{\addtocontents{toc}{\protect\begin{multicols}{2}}}
% \AtEndDocument  {\addtocontents{toc}{\protect\end  {multicols}}}
% \end{verbatim}
% \vspace{-2em}
% 这种写入 |.toc| 文件的方法对于试图控制目录的排版输出(typesetting output)是最灵活的。
%
% 为使 \Lpack{tocloft}\ {\kaiti 稍}简单些,可以使用以下宏:\cs{cftZprehook}\ 和 \cs{cftZprehook},
% 其中 |Z| 是 |toc|、|lof|、|lot| 等。如果定义了这些宏,它们会在目录条目的实际排版之前
% 插入材料(material),以此类推。因此,可以实现多栏目录(multicolumn ToCs):
% \begin{verbatim}
% \RequirePackage{multicol}
% \renewcommand\cfttocprehook{\begin{multicols}{2}}
% \renewcommand\cfttocposthook{\end{multicols}}
% \end{verbatim}
%
% \subsection[与其他宏包一起使用]{\heiti 与其他宏包一起使用}
%
% \Lpack{tocloft}\ 和 \Lpack{tocbibind}\ 宏包可以在同一文档中一起使用。
% \Lpack{tocbibind}\ 宏包提供了向目录(Table of Contents)添加如参考文献(bibliography)
% 和索引(index)这样的文档元素(document elements)的简单方法。然而,
% 有一个已知的潜在问题(known potential problem):
% \begin{itemize}
% \item 如果 |\tocotherhead| 命令的参数不是一个正常的节划分(sectioning divisions)
% (即从部分[part]到子段[sub-paragraph]),比如 |\tocotherhead{clause}|,那么这几乎
% 肯定会导致问题(因为 \Lpack{tocloft}\ 宏包不知道如何定义相应的 |\l@clause| 命令)。
% 在这种情况下,您必须自己提供适当的宏。
% \end{itemize}
%
% 有些宏包,比如 Anselm Lingnau (安塞姆·林瑙) 的 \Lpack{float}\ 宏包,可以创建
% 其他类型的 \textit{List of \ldots}。\Lpack{tocloft}\ 宏包只能最小限度地改变这些包的格式,
% 主要是因为这些宏包彼此独立,而且,在 \Lpack{float}\ 宏包的情况下,可以在文档的任何位置
% 动态创建新类型的 float 环境及其相关列表(lists)。排版的某些方面
% 由 |\@cftbsnum|、|\@cftasnum| 和 |\@cftasnumb| 命令控制。
% \DescribeMacro{\@cftbsnum}
% \DescribeMacro{\@cftasnum}
% \DescribeMacro{\@cftasnumb}
% 这些命令等效于前面描述的 |\cftXpresnum|、|\cftXaftersnum| 和 |\cftXaftersnumb| 命令。在默认情况下,
% 它们被定义为什么也不做,但可能会被重新定义为做某事情。
%
% \Lpack{tocloft}\ 宏包和 \Lpack{minitoc}\ 宏包有一个不幸的
% 交互(unfortunate interaction)\ \footnote{发现者是林登·杜丁(\texttt{lyndon.dudding@totalise.co.uk})。},
% 幸运的是它可以被修复。在正常情况下,当在分章文档(chaptered document)中使用 \Lpack{minitoc}\ 时,
% 它会将 minitocs 中的节条目(section entries)以粗体字体(bold font)排版。
% 如果 \Lpack{tocloft}\ 与 Lpack{minitoc}\ 结合使用,则 minitocs 节条目(section entries)
% 将以普通字体(normal font)排版,但页码(page numbers)为粗体字体,而目录(ToC)中的
% 节条目(section entries)则全部为普通字体。
%
% 如果您希望 minitoc 节条目(section entries)全部使用普通字体(normal font),
% 一个解决方法是在导言区(preamble)放置将下述命令:
% \begin{verbatim}
% \renewcommand{\mtcSfont}{\small\normalfont}
% \end{verbatim}
% \vspace{-1.5em}
% 否则,解决方法就是下面的魔法(incantation):
% \begin{verbatim}
% \renewcommand{\cftsecfont}{\bfseries}
% \renewcommand{\cftsecleader}{\bfseries\cftdotfill{\cftdotsep}}
% \renewcommand{\cftsecpagefont}{\bfseries}
% \end{verbatim}
% \vspace{-1.5em}
% 要将 ToC 和 minitoc 中的节条目(section entries)都加粗,请在导言中加入咒语(incantation)。
% 若要使 minitoc 节条目(section entries)以粗体显示,而 ToC 条目以普通字体(normal font)显示,
% 请将上述魔法放在 |\tableofcontents| 命令和第一个 |\chapter| 命令之间。
%
%
% 一般来说,与其他重新定义的宏包(而该宏包也能被 \Lpack{tocloft}\ 修改)一起使用可能会有问题。
%
% \clearpage
% \section{\heiti 宏包代码} \label{sec:code}
%
%    \begin{macrocode}
%<*usc>
%    \end{macrocode}
%
% 为了尽量避免与其他包发生名称冲突(name clashes),每个内部名称(internal name)都将
% 包含字符串 \texttt{@cft}。
%
% \begin{macro}{\@cftifundefined}
% \changes{v2.3h}{2017/08/29}{Use internal ifundefined macro to fix conflict with fancyhdr}
% 由于与这个宏包和 fancyhdr 在如何检查未定义(checked for undefinedness)上会发生的冲突。
%    \begin{macrocode}
\newcommand\@cftifundefined[1]{%
  \begingroup\expandafter\expandafter\expandafter\endgroup
  \expandafter\ifx\csname #1\endcsname\relax
    \expandafter\@firstoftwo
  \else
    \expandafter\@secondoftwo
  \fi}
%    \end{macrocode}
% \end{macro}
%
% \begin{macro}{\@cftquit}
% \begin{macro}{\if@cfthaschapter}
% 我们将为 ToC 等使用章(chapter)或节(section)类型的标题(type headings),因此我们
% 需要知道文档类(document class)支持哪一种。
% \changes{v2.0}{2001/03/03}{Revamped chapter checking so the stdclsdv package no longer required}
%    \begin{macrocode}
\newcommand{\@cftquit}{}
\newif\if@cfthaschapter
%    \end{macrocode}
% \end{macro}
% \end{macro}
%
% \begin{macro}{\if@cftkoma}
% \Lpack{koma}\ 类的默认值与标准类(standard classes)不同,因此我们需要知道
% 是否加载了 \Lpack{koma}\ 类。
% \changes{v2.3}{2002/06/15}{Added check for a koma class}
%    \begin{macrocode}
\newif\if@cftkoma
\@cftkomafalse
\@ifclassloaded{scrartcl}{\@cftkomatrue}{}
\@ifclassloaded{scrreprt}{\@cftkomatrue}{}
\@ifclassloaded{scrbook}{\@cftkomatrue}{}
%    \end{macrocode}
% \end{macro}
%
% \begin{macro}{\if@cfttitlesec}
% \changes{v2.3f}{2013/04/30}{Added check for titlesec package}
%    \begin{macrocode}
\newif\if@cfttitlesec
\AtBeginDocument{\@ifpackageloaded{titlesec}{\@cfttitlesectrue}{}}
%    \end{macrocode}
% \end{macro}
%
% 如果没有可识别的节分区(recognised sectional divisions),则发出警告,然后跳过宏包代码的其余部分。
%    \begin{macrocode}
\@cftifundefined{chapter}{%
  \@cfthaschapterfalse
  \@cftifundefined{section}{%
    \PackageWarning{tocloft}%
      {I don't recognize any sectional divisions so I'll do nothing}
    \renewcommand{\@cftquit}{\endinput}
    }{\PackageInfo{tocloft}{The document has section divisions}}
  }{\@cfthaschaptertrue
    \PackageInfo{tocloft}{The document has chapter divisions}}
%    \end{macrocode}
% 也许现在就退出。
%    \begin{macrocode}
\@cftquit
%    \end{macrocode}
%
% 如果 |\if@cfthaschapter|\ 为 TRUE,则使用章样式(chapter style),否则使用节样式(section style)。
%
% \begin{macro}{\if@cfttocbibind}
% 如果 \Lpack{tocbibind}\ 宏包已经加载,则标志(flag)设置为 TRUE。1998/11/15 版本
% 的 \Lpack{tocbibind}\ 不一定适用于 \Lpack{tocloft}。
%    \begin{macrocode}
\newif\if@cfttocbibind
\AtBeginDocument{%
  \@ifpackageloaded{tocbibind}{\@cfttocbibindtrue}{\@cfttocbibindfalse}
  \if@cfttocbibind
    \@ifpackagelater{tocbibind}{1998/11/16}{}{%
      \PackageWarning{tocloft}{%
You are using a version of the tocbibind package\MessageBreak
that is not compatible with tocloft.\MessageBreak
The results may be surprising.\MessageBreak
Consider installing the current version of tocbibind.}}
  \fi
}
%    \end{macrocode}
% \end{macro}
%
% \begin{macro}{\if@cftnctoc}
% 用于实现 \Lopt{titles}\ 选项的布尔值(boolean)。如果目录(ToC)、表格目录(LoT)、图形目录(LoF)标题应该使用
% 默认样式(default styles),则为 TRUE。
%    \begin{macrocode}
\newif\if@cftnctoc\@cftnctocfalse
\DeclareOption{titles}{\@cftnctoctrue}
  %% \ProcessOptions\relax
%    \end{macrocode}
% \end{macro}
%
% \begin{macro}{\if@cftsubfigopt}
% 用于实现 \Lopt{subfigure}\ 选项的布尔值。
%    \begin{macrocode}
\newif\if@cftsubfigopt\@cftsubfigoptfalse
\DeclareOption{subfigure}{\@cftsubfigopttrue}
%    \end{macrocode}
% \end{macro}
%
% 处理选项。
%
%    \begin{macrocode}

\ProcessOptions\relax

%    \end{macrocode}
%
% \begin{macro}{\tocloftpagestyle}
% \begin{macro}{\@cftpagestyle}
% 用户级宏(user-level macro),用于设置 To C的第一页的页面样式(pagestyle)等。
% 默认为 |plain| 页面样式。
% \changes{v2.3}{2002/06/15}{Added \cs{tocloftpagestyle}}
%    \begin{macrocode}
\newcommand{\tocloftpagestyle}[1]{%
  \def\@cftpagestyle{\thispagestyle{#1}}}
\tocloftpagestyle{plain}

%    \end{macrocode}
% \end{macro}
% \end{macro}
%
% \begin{macro}{\cftmarktoc}
% \begin{macro}{\cftmarklof}
% \begin{macro}{\cftmarklot}
% 这三个宏设置了页眉标题(running heads)的样式。它们被初始化为默认的外观(default appearance)。
% \changes{v2.3}{2002/06/15}{Marking commands are different for koma}
%    \begin{macrocode}
\newcommand{\cftmarktoc}{%
  \@mkboth{\MakeUppercase\contentsname}{\MakeUppercase\contentsname}}
\newcommand{\cftmarklof}{%
  \@mkboth{\MakeUppercase\listfigurename}{\MakeUppercase\listfigurename}}
\newcommand{\cftmarklot}{%
  \@mkboth{\MakeUppercase\listtablename}{\MakeUppercase\listtablename}}
\if@cftkoma
  \renewcommand{\cftmarktoc}{%
    \@mkboth{\contentsname}{\contentsname}}
  \renewcommand{\cftmarklof}{%
    \@mkboth{\listfigurename}{\listfigurename}}
  \renewcommand{\cftmarklot}{%
    \@mkboth{\listtablename}{\listtablename}}
\fi
%    \end{macrocode}
% \end{macro}
% \end{macro}
% \end{macro}
%
% \begin{macro}{\@cfttocstart}
% \begin{macro}{\@cfttocfinish}
% 在 |\tableofcontents| 命令的开头和结尾执行操作(perform the actions)的两个宏
% (和好友)。|\@cfttocstart| 处理分章文档(chaptered documents),确保 ToC 排版成
% 单栏(single column)(原始代码请参见 \file{classes.dtx})。\Lpack{ccaption}\ 宏包也提供这些宏。
%    \begin{macrocode}
\providecommand{\@cfttocstart}{%
  \if@cfthaschapter
    \if@twocolumn
      \@restonecoltrue\onecolumn
    \else
      \@restonecolfalse
    \fi
  \fi}
%    \end{macrocode}
% 如果需要,|\@cfttocfinish| 重置为两栏排版。
%    \begin{macrocode}
\providecommand{\@cfttocfinish}{%
  \if@cfthaschapter
    \if@restonecol\twocolumn\fi
  \fi}
%    \end{macrocode}
% \end{macro}
% \end{macro}
%
% \begin{macro}{\phantomsection}
% 这是因为 \Lopt{hyperref}\ 宏包与 |\addcontentsline|\ 一起使用。
% \changes{v2.2}{2001/04/17}{Provided \cs{phantomsection}}
% \changes{v2.2}{2001/04/17}{Added \cs{phantomsection} before \cs{addcontentsline} commands}
%    \begin{macrocode}
\providecommand{\phantomsection}{}

%    \end{macrocode}
% \end{macro}
%
% \begin{macro}{\@cftdobibtoc}
% 如果使用了 \Lpack{tocbibind}\ 宏包,并且它重新定义了 |\tableofcontents|,
% 我们需要满足这一点。定义的内容(contents of the definition)在 \Lpack{tocbibind}\ 中定义。
%    \begin{macrocode}
\newcommand{\@cftdobibtoc}{%
  \if@dotoctoc
    \if@bibchapter
      \phantomsection
      \addcontentsline{toc}{chapter}{\contentsname}
    \else
      \phantomsection
      \addcontentsline{toc}{\@tocextra}{\contentsname}
    \fi
  \fi}

%    \end{macrocode}
% \end{macro}
%
% \begin{macro}{\cftparskip}
% ToC 等的局部 |\parskip| 设置为长度(length) |\cftparskip|。
% \changes{v2.1}{2001/04/08}{Added \cs{cftparskip} for local parskip in ToC, etc.}
%    \begin{macrocode}
\newlength{\cftparskip}
\setlength{\cftparskip}{0pt}

%    \end{macrocode}
% \end{macro}
%
%
% \begin{macro}{\tableofcontents}
% 这是默认 |\tableofcontents| 命令的参数化版本(parameterised version)。
% 每个类(class)都有自己的定义,但我们必须在一个定义中满足所有类,因此需要进行一些检查(checks)。
% 在加载所有宏包之后修改定义。
%
% 如果已使用 \Lopt{titles}\ 选项,则不会修改该命令。
%
%    \begin{macrocode}
\AtBeginDocument{%
\if@cftnctoc
  % ensure \cftparskip is still set properly
  \let\OLD@starttoc\@starttoc
  \renewcommand{\@starttoc}[1]{%
    \begingroup
      \parskip=\cftparskip
      \OLD@starttoc{#1}%
    \endgroup
  }
\else
  \renewcommand{\tableofcontents}{%
    \@cfttocstart
%    \end{macrocode}
% 确保前面的段落(previous paragraph)已经完成。在一个组(group)中设置局部段落样式
% (local paragraphing style)并排版标题(title)。
%    \begin{macrocode}
    \par
    \begingroup
      \parindent\z@ \parskip\cftparskip
      \@cftmaketoctitle
%    \end{macrocode}
% 如果已使用 \Lpack{tocbibind},则将 ToC 名称添加到 ToC
%    \begin{macrocode}
      \if@cfttocbibind
        \@cftdobibtoc
      \fi
%    \end{macrocode}
% 最后,读取 \file{.toc}\ 文件并结束(finish up)。
%    \begin{macrocode}
      \@starttoc{toc}%
    \endgroup
    \@cfttocfinish}
\fi
}
%    \end{macrocode}
% \end{macro}
%
% \begin{macro}{\@cftmaketoctitle}
% 此命令排版 ToC 的标题。
% \changes{v2.3}{2002/06/15}{Added \cs{@secpenalty} to \cs{@cftmaketoctitle}}
% \changes{v2.3}{2002/06/15}{Added \cs{@cftpagestyle} to \cs{@cftmaketoctitle}}
%    \begin{macrocode}
\newcommand{\@cftmaketoctitle}{%
  \addpenalty\@secpenalty
  \if@cfthaschapter
    \vspace*{\cftbeforetoctitleskip}%
  \else
    \vspace{\cftbeforetoctitleskip}%
  \fi
  \@cftpagestyle
  {\interlinepenalty\@M
  {\cfttoctitlefont\contentsname}{\cftaftertoctitle}%
  \cftmarktoc
  \par\nobreak
  \vskip \cftaftertoctitleskip
  \@afterheading}}
%    \end{macrocode}
% \end{macro}
%
% \begin{macro}{\cftbeforetoctitleskip}
% \begin{macro}{\cftaftertoctitleskip}
%  这两个长度(lengths)控制 ToC 标题前后的垂直间距(vertical spacing)。
%    \begin{macrocode}
\newlength{\cftbeforetoctitleskip}
\newlength{\cftaftertoctitleskip}
%    \end{macrocode}
% 它们的值取决于文档是否有章(chapters)。在分章文档(chaptered documents)中,默认
% 的 ToC 标题被排版为 |\chapter*|,否则排版为为 |\section*|。
%    \begin{macrocode}
\if@cfthaschapter
  \setlength{\cftbeforetoctitleskip}{50pt}
  \setlength{\cftaftertoctitleskip}{40pt}
\else
  \setlength{\cftbeforetoctitleskip}{3.5ex \@plus 1ex \@minus .2ex}
  \setlength{\cftaftertoctitleskip}{2.3ex \@plus.2ex}
\fi
%    \end{macrocode}
% \end{macro}
% \end{macro}
%
% \begin{macro}{\cfttoctitlefont}
% \begin{macro}{\cftaftertoctitle}
% ToC 标题按 |\cfttoctitlefont| 给定的样式排版。在排版标题后调用 |\cftaftertoctitle| 宏。
% 这被初始化为不执行任何操作。这两个宏都可以被重新定义以执行其他操作(例如,在 |\cfttoctitlefont| 中
% 添加一个 |\hfill| 会使标题右对齐[flushright])。
% \changes{v2.3}{2002/06/15}{koma changes the title fonts}
%    \begin{macrocode}
\if@cfthaschapter
  \newcommand{\cfttoctitlefont}{\normalfont\Huge\bfseries}
  \if@cftkoma\renewcommand{\cfttoctitlefont}{\size@chapter\sectfont}\fi
\else
  \newcommand{\cfttoctitlefont}{\normalfont\Large\bfseries}
  \if@cftkoma\renewcommand{\cfttoctitlefont}{\size@section\sectfont}\fi
\fi
\newcommand{\cftaftertoctitle}{}
%    \end{macrocode}
% \end{macro}
% \end{macro}
%
% \begin{macro}{\cftsetpnumwidth}
% \begin{macro}{\cftsetrmarg}
%  用于设置 |\@pnumwidth| 和 |\@tocrmarg| 的用户命令。
%    \begin{macrocode}
\newcommand{\cftsetpnumwidth}[1]{\renewcommand{\@pnumwidth}{#1}}
\newcommand{\cftsetrmarg}[1]{\renewcommand{\@tocrmarg}{#1}}
%    \end{macrocode}
% \end{macro}
% \end{macro}
%
% \begin{macro}{\cftpnumalign}
% \changes{v2.3f}{2013/05/02}{Add \cs{cftpnumalign} for page number alignment}
% 对齐字符串(作为页码盒子的 \cs{makebox}\ 输入)。
%    \begin{macrocode}
\newcommand{\cftpnumalign}{r}
%    \end{macrocode}
% \end{macro}
%
% \begin{macro}{\cftdot}
% \begin{macro}{\cftdotfill}
% 在默认 ToC 中,可以使用点线(dotted line)在标题(title)和页码(page number)之间提供引线(leader)。
% 该引线的定义隐藏在 |\@dottedtocline| 命令中。|\cftdotfill{|\meta{sep}|}| 命令提供
% 引线代码(leader code)的参数化版本(parameterised version),其中 \meta{sep}\ 是以 mu 为
% 单位的点(dots)之间的间距(separation)。引线中用于“点(dots)”的符号(symbol)由 |\cftdot| 的值给出。
% 这些宏也由 \Lpack{ccaption}\ 宏包提供。
%    \begin{macrocode}
\providecommand{\cftdot}{.}
\providecommand{\cftdotfill}[1]{%
  \def\@tempa{#1}%
  \def\@tempb{\cftnodots}%
  \ifx\@tempa\@tempb
    \hfill
  \else
    \leaders\hbox{$\m@th\mkern #1 mu\hbox{\cftdot}\mkern #1 mu$}\hfill
  \fi
}
%    \end{macrocode}
% \changes{v2.3f}{2013/04/30}{Specifically avoid placing dots when \cs{cftnodots} is used}
% \end{macro}
% \end{macro}
%
% \begin{macro}{\cftdotsep}
% \begin{macro}{\cftnodots}
% |\cftdotsep| 保留默认的点分隔(default dot separation),也由 \Lpack{ccaption}\ 包提供。
% 如果 |\cftdotfill| 中的压缩字距(kerns)足够大,则不会打印任何点。|\cftnodots| 应“足够大”。
% (实际上,|\cftnodots| 现在用作条件分支[conditional branch]的标志[flag],因此其数值现在不再那么重要。)
%    \begin{macrocode}
\providecommand{\cftdotsep}{4.5}
\newcommand{\cftnodots}{5000}
%    \end{macrocode}
% \changes{v2.3f}{2013/04/30}{Reduce \cs{cftnodots} to avoid very large dotsep sizes.}
% \end{macro}
% \end{macro}
%
% 现在来看一下关于 ToC 条目排版的更棘手的问题。
%
% \file{.toc}(也是 \file{.lof}\ 和 \file{.lot})文件是由 |\contentsline{|\meta{kind}|}{|\meta{title}|}{|\meta{page}|}| 命
% 令组成的列表(list),其中 \meta{kind}\ 是标题的类型(kind of heading)(例如 |part| 或 |section| 或 |figure|)。
% \meta{title}\ 是标题文本(title text)(包括编号[number])。\meta{page}\ 是页码(page number)。通过
% 调用 |\addcontentsline{|\meta{file}|}{|\meta{kind}|}{|\meta{title}|}| 命令将条目(entries)插入文件,
% 其中 \meta{file}\ 是文件扩展名(例如|toc|、|lot|),其他参数与 |\contentsline| 命令相同。
% 任意内容也可以通过 |\addtocontents{|\meta{file}|}{|\meta{text}|}| 命令放入文件中)。
% |\contentsline| 条目的排版由 |\l@kind| 形式(form)的命令执行。分节命令(sectioning commands)和
% 标题命令(captioning commands)调用 |\addcontentsline| 将标题(titles)插入 \file{.toc}\ 等文件中。
%
% 为了方便起见,通常不可能将标题(title)和其编号(number)的排版分开处理,因为两者
% 都绑定到(bundled into) |\contentsline| 中的 \meta{title}\ 参数。如果适当修改
% 了 |\contentsline| 命令,则可以单独处理它们。如果这样做了,那么 |\addtocontentsline| 命令也需要修改,
% 同样也需要修改分节命令(sectioning commands)和标题命令(captioning commands)。这当然是可能的,
% 但如果任何其他宏包也修改了分节命令或标题命令,就会产生问题,并且有几个宏包可以这样做。
%
% 如上所述,对于除 Part 条目外的所有条目,节编号(sectional number)都是通过 |\numberline| 命令进行排版的。
% 我们可以利用这个事实。
%
% 我已经决定不触碰 |\contentsline| 宏,而是在它存在的情况下使用它。也就是说,我将修改 |\l@kind| 命令。
% 本质上,我的新定义由 |\@dottedtocline| 代码的内联版本(inlined versions)组成。
%
% \begin{macro}{\cftparfillskip}
% |\l@kind| 命令(局部[locally])修改 |\parfillskip| 的值。|\cftparfillskip| 是 \textit{\TeX book} |\parfillskip| 默认定义的副本。
%    \begin{macrocode}
\newcommand{\cftparfillskip}{\parfillskip=0pt plus1fil}
%    \end{macrocode}
% \end{macro}
%
% \begin{macro}{\numberline}
% \changes{v1.1}{2000/02/11}{Added \cs{@cftbsnum} to \cs{numberline}}
%  |\numberline{|\meta{secnum}|}| 命令的目的是在一个宽度为 |\@tempdima| 的盒子中
%  左对齐排版 \meta{secnum}。我重新定义了它,添加了三个额外的参数(additional parameters),
%  即 |\@cftbsnum|、|\@cftasnum| 和 |\@cftasnumb| (原始定义见 \file{ltsect.dtx})。
%    \begin{macrocode}
\renewcommand{\numberline}[1]{%
  \hb@xt@\@tempdima{\@cftbsnum #1\@cftasnum\hfil}\@cftasnumb}
%    \end{macrocode}
% \end{macro}
%
% \begin{macro}{\@cftbsnum}
% \begin{macro}{\@cftasnum}
% \begin{macro}{\@cftasnumb}
% \changes{v0.2b}{1999/03/07}{Added empty definitions for @cftasnum and @cftasnumb commands}
% \changes{v1.1}{2000/02/11}{Added empty definition of \cs{@cftbsnum}}
% 最初,这些命令未定义,但是在 |\l@...| 命令中 |\let| 使用适当的命令,但是必须定义它们,
% 以防意外调用 |\numberline|,例如通过使用 \Lpack{float}\ 宏包\ \footnote{这个 bug 是 Andrew Thurber (安德鲁·瑟伯)
% 在同时使用 \Lpack{tocloft}\ 宏和 \Lpack{algorithm}\ 宏包时发现的。}。
%    \begin{macrocode}
\newcommand{\@cftbsnum}{}
\newcommand{\@cftasnum}{}
\newcommand{\@cftasnumb}{}
%    \end{macrocode}
% \end{macro}
% \end{macro}
% \end{macro}
%
%
%
% \begin{macro}{\l@part}
% \begin{macro}{\if@cftdopart}
% \changes{v1.1}{2000/02/11}{Added \cs{@cftbsnum} and \cs{cftXpresnum} to
%                            all \cs{\l@X} commands}
% \changes{v1.1}{2000/02/11}{Added \cs{cftXpresnum} commands}
% |\l@part{|\meta{title}|}{|\meta{page}|}| 排版 |part| 标题(heading)的 ToC 条目。
% 它是默认 |\l@part| 的参数化副本(parameterised copy)(参见 \file{classes.dtx}\ 以
% 获取原始定义,并获取 |\l@subsection| 的以下代码及大部分代码的解释)。默认情况下,
% 部分( Parts)(和章[Chapters])没有点引线(dotted leaders)。此包为所有条目提供了点引线。
% \changes{v0.2a}{1999/01/24}{In article class, Part level is 0 not -1}
% \changes{v2.0}{2001/03/03}{Checked directly for \cs{part} definition}
% \changes{v2.3a}{2002/10/03}{Added \cs{cftpartpresnum} to \cs{l@part}}
%
%    \begin{macrocode}
\newif\if@cftdopart
\newif\if@cfthaspart
\@cftifundefined{part}{\@cfthaspartfalse}{\@cfthasparttrue}
\if@cfthaspart
\renewcommand*{\l@part}[2]{%
  \@cftdopartfalse
  \ifnum \c@tocdepth >-2\relax
    \if@cfthaschapter
      \@cftdoparttrue
    \fi
    \ifnum \c@tocdepth >\m@ne
      \if@cfthaschapter\else
        \@cftdoparttrue
      \fi
    \fi
  \fi
  \if@cftdopart
    \if@cfthaschapter
      \addpenalty{-\@highpenalty}%
    \else
      \addpenalty\@secpenalty
    \fi
    \addvspace{\cftbeforepartskip}%
    \begingroup
      {\leftskip \cftpartindent\relax
       \rightskip \@tocrmarg
       \parfillskip -\rightskip
       \parindent \cftpartindent\relax\@afterindenttrue
       \interlinepenalty\@M
       \leavevmode
       \@tempdima \cftpartnumwidth\relax
       \let\@cftbsnum \cftpartpresnum
       \let\@cftasnum \cftpartaftersnum
       \let\@cftasnumb \cftpartaftersnumb
       \advance\leftskip \@tempdima \null\nobreak\hskip -\leftskip
%    \end{macrocode}
% 在默认的 \LaTeX\ 中,一部分 ToC 条目在写入时没有 \verb|\numberline|,因此需要
% 手动插入“presnum”。在 Koma-Script (科马脚本)和 titlesec (可能还有其他地方---请让我知晓!)中,然而,情况并非如此。
%    \begin{macrocode}
       {\cftpartfont \if@cftkoma\else\if@cfttitlesec\else\cftpartpresnum\fi\fi #1}%
       \cftpartfillnum{#2}}
      \nobreak
      \if@cfthaschapter
        \global\@nobreaktrue
        \everypar{\global\@nobreakfalse\everypar{}}%
      \else
        \if@compatibility
          \global\@nobreaktrue
          \everypar{\global\@nobreakfalse\everypar{}}%
        \fi
      \fi
    \endgroup
  \fi}
\fi
%    \end{macrocode}
% \end{macro}
% \end{macro}
%
% \begin{macro}{\cftbeforepartskip}
% \begin{macro}{\cftpartnumwidth}
% \begin{macro}{\cftpartfont}
% \begin{macro}{\cftpartpresnum}
% \begin{macro}{\cftpartaftersnum}
% \begin{macro}{\cftpartaftersnumb}
% \begin{macro}{\cftpartleader}
% \begin{macro}{\cftpartdotsep}
% \begin{macro}{\cftpartpagefont}
% \begin{macro}{\cftpartafterpnum}
% \begin{macro}{\cftpartindent}
% \begin{macro}{\cftpartfillnum}
% 这些是用于控制部分条目(Part entries)排版的用户命令(user commands)。它们经过了初始化,
% 以提供标准外观(standard appearance)。
% \changes{v2.3a}{2002/10/03}{Deleted \cs{cftpartaftersnum} and \cs{cftpartaftersnumb}}
% \changes{v2.3b}{2003/01/20}{Reinstated \cs{cftpartaftersnum} and \cs{cftpartaftersnumb}}
%    \begin{macrocode}
\if@cfthaspart
  \newlength{\cftbeforepartskip}
    \setlength{\cftbeforepartskip}{2.25em \@plus\p@}
  \newlength{\cftpartnumwidth}
    \setlength{\cftpartnumwidth}{0em}
  \newcommand{\cftpartfont}{\large\bfseries}
  \newcommand{\cftpartpresnum}{}
  \newcommand{\cftpartaftersnum}{}
  \newcommand{\cftpartaftersnumb}{}
  \newcommand{\cftpartleader}{\large\bfseries\cftdotfill{\cftpartdotsep}}
  \newcommand{\cftpartdotsep}{\cftnodots}
  \newcommand{\cftpartpagefont}{\large\bfseries}
  \newcommand{\cftpartafterpnum}{}
  \newlength{\cftpartindent}
    \setlength{\cftpartindent}{0em}
  \newcommand{\cftpartfillnum}[1]{%
    {\cftpartleader}%
    {\makebox[\@pnumwidth][\cftpnumalign]{
        \cftpartpagefont #1}\cftpartafterpnum\par}%
  }
%    \end{macrocode}
% \Lpack{koma}\ 类使用一些不同的设置。
% \changes{v2.3}{2002/06/15}{koma has different part settings}
%    \begin{macrocode}
  \if@cftkoma
    \setlength{\cftpartnumwidth}{2em}
    \renewcommand{\cftpartfont}{\sectfont\large}
    \renewcommand{\cftpartpagefont}{\sectfont\large}
  \fi
\fi
%    \end{macrocode}
% \end{macro}
% \end{macro}
% \end{macro}
% \end{macro}
% \end{macro}
% \end{macro}
% \end{macro}
% \end{macro}
% \end{macro}
% \end{macro}
% \end{macro}
% \end{macro}
%
% \begin{macro}{\l@chapter}
% |\l@chapter{|\meta{title}|}{|\meta{page}|}| 排版章标题(|chapter| heading)的 ToC 条目。
% 它是默认 |\l@chapter| 的参数化副本(parameterised copy)(参见 \file{classes.dtx}\ 获
% 取原始定义[original definition])。这仅适用于分章文档(chaptered documents)。
%    \begin{macrocode}
\if@cfthaschapter
\renewcommand*{\l@chapter}[2]{%
  \ifnum \c@tocdepth >\m@ne
    \addpenalty{-\@highpenalty}%
    \vskip \cftbeforechapskip
    {\leftskip \cftchapindent\relax
     \rightskip \@tocrmarg
     \parfillskip -\rightskip
     \parindent \cftchapindent\relax\@afterindenttrue
     \interlinepenalty\@M
     \leavevmode
     \@tempdima \cftchapnumwidth\relax
     \let\@cftbsnum \cftchappresnum
     \let\@cftasnum \cftchapaftersnum
     \let\@cftasnumb \cftchapaftersnumb
     \advance\leftskip \@tempdima \null\nobreak\hskip -\leftskip
     {\cftchapfont #1}\nobreak
     \cftchapfillnum{#2}}%
  \fi}%
\fi
%    \end{macrocode}
% \end{macro}
%
% \begin{macro}{\cftbeforechapskip}
% \begin{macro}{\cftchapindent}
% \begin{macro}{\cftchapnumwidth}
% \begin{macro}{\cftchapfont}
% \begin{macro}{\cftchappresnum}
% \begin{macro}{\cftchapaftersnum}
% \begin{macro}{\cftchapaftersnumb}
% \begin{macro}{\cftchapleader}
% \begin{macro}{\cftchapdotsep}
% \begin{macro}{\cftchappagefont}
% \begin{macro}{\cftchapafterpnum}
% \begin{macro}{\cftchapfillnum}
% 这些是控制章条目(Chapter entries)排版的用户命令。它们经过了初始化,以提供标准外观(standard appearance)。
%    \begin{macrocode}
\if@cfthaschapter
  \newlength{\cftbeforechapskip}
    \setlength{\cftbeforechapskip}{1.0em \@plus\p@}
  \newlength{\cftchapindent}
    \setlength{\cftchapindent}{0em}
  \newlength{\cftchapnumwidth}
    \setlength{\cftchapnumwidth}{1.5em}
  \newcommand{\cftchapfont}{\bfseries}
  \newcommand{\cftchappresnum}{}
  \newcommand{\cftchapaftersnum}{}
  \newcommand{\cftchapaftersnumb}{}
  \newcommand{\cftchapleader}{\bfseries\cftdotfill{\cftchapdotsep}}
  \newcommand{\cftchapdotsep}{\cftnodots}
  \newcommand{\cftchappagefont}{\bfseries}
  \newcommand{\cftchapafterpnum}{}
  \newcommand{\cftchapfillnum}[1]{%
    {\cftchapleader}\nobreak
    \makebox[\@pnumwidth][\cftpnumalign]{
        \cftchappagefont #1}\cftchapafterpnum\par
  }
%    \end{macrocode}
% \Lpack{koma}\ 类有不同的章设置(chapter settings)。
% \changes{v2.3}{2002/06/15}{koma has different chapter settings}
%    \begin{macrocode}
  \if@cftkoma
    \renewcommand{\cftchapfont}{\sectfont}
  \fi
\fi

%    \end{macrocode}
% \end{macro}
% \end{macro}
% \end{macro}
% \end{macro}
% \end{macro}
% \end{macro}
% \end{macro}
% \end{macro}
% \end{macro}
% \end{macro}
% \end{macro}
% \end{macro}
%
% \begin{macro}{\l@section}
% |\l@section{|\meta{title}|}{|\meta{page}|}| 排版节标题(|section| heading)的 ToC 条目。
% 它是默认 |\l@section| 的参数化副本(parameterised copy)(请参见 \file{classes.dtx}\ 以获取原始定义)。
%    \begin{macrocode}
\renewcommand*{\l@section}[2]{%
  \ifnum \c@tocdepth >\z@
    \if@cfthaschapter
      \vskip \cftbeforesecskip
    \else
      \addpenalty\@secpenalty
      \addvspace{\cftbeforesecskip}
    \fi
    {\leftskip \cftsecindent\relax
     \rightskip \@tocrmarg
     \parfillskip -\rightskip
     \parindent \cftsecindent\relax\@afterindenttrue
     \interlinepenalty\@M
     \leavevmode
     \@tempdima \cftsecnumwidth\relax
     \let\@cftbsnum \cftsecpresnum
     \let\@cftasnum \cftsecaftersnum
     \let\@cftasnumb \cftsecaftersnumb
     \advance\leftskip \@tempdima \null\nobreak\hskip -\leftskip
     {\cftsecfont #1}\nobreak
     \cftsecfillnum{#2}}%
  \fi}
%    \end{macrocode}
% \end{macro}
%
% \begin{macro}{\cftbeforesecskip}
% \begin{macro}{\cftsecindent}
% \begin{macro}{\cftsecnumwidth}
% \begin{macro}{\cftsecfont}
% \begin{macro}{\cftsecpresnum}
% \begin{macro}{\cftsecaftersnum}
% \begin{macro}{\cftsecaftersnumb}
% \begin{macro}{\cftsecleader}
% \begin{macro}{\cftsecdotsep}
% \begin{macro}{\cftsecpagefont}
% \begin{macro}{\cftsecafterpnum}
% \begin{macro}{\cftsecfillnum}
% 这些是用户命令,用于控制节条目(Section entries)的排版。它们经过了初始化,
% 以提供标准外观(standard appearance)。
%    \begin{macrocode}
\newlength{\cftbeforesecskip}
\newlength{\cftsecindent}
\newlength{\cftsecnumwidth}
\newcommand{\cftsecpresnum}{}
\newcommand{\cftsecaftersnum}{}
\newcommand{\cftsecaftersnumb}{}
\if@cfthaschapter
  \setlength{\cftbeforesecskip}{\z@ \@plus.2\p@}
  \setlength{\cftsecindent}{1.5em}
  \setlength{\cftsecnumwidth}{2.3em}
  \newcommand{\cftsecfont}{\normalfont}
  \newcommand{\cftsecleader}{\normalfont\cftdotfill{\cftsecdotsep}}
  \newcommand{\cftsecdotsep}{\cftdotsep}
  \newcommand{\cftsecpagefont}{\normalfont}
\else
  \setlength{\cftbeforesecskip}{1.0em \@plus\p@}
  \setlength{\cftsecindent}{0em}
  \setlength{\cftsecnumwidth}{1.5em}
  \newcommand{\cftsecfont}{\bfseries}
  \newcommand{\cftsecleader}{\bfseries\cftdotfill{\cftsecdotsep}}
  \newcommand{\cftsecdotsep}{\cftnodots}
  \newcommand{\cftsecpagefont}{\bfseries}
\fi
\newcommand{\cftsecafterpnum}{}
\newcommand{\cftsecfillnum}[1]{%
  {\cftsecleader}\nobreak
  \makebox[\@pnumwidth][\cftpnumalign]{
    \cftsecpagefont #1}\cftsecafterpnum\par
}
%    \end{macrocode}
% \end{macro}
% \end{macro}
% \end{macro}
% \end{macro}
% \end{macro}
% \end{macro}
% \end{macro}
% \end{macro}
% \end{macro}
% \end{macro}
% \end{macro}
% \end{macro}
%
% \begin{macro}{\l@subsection}
% |\l@subsection{|\meta{title}|}{|\meta{page}|}| 排版子节标题(|subsection| heading)的 ToC 条目。
% 它是默认 |\l@subsection| 的参数化副本(parameterised copy)(请参见 \file{classes.dtx}\ 以获取原始定义)。
%    \begin{macrocode}
\renewcommand*{\l@subsection}[2]{%
%    \end{macrocode}
% 只有当条目在 |tocdepth| 内时,才可以对其进行排版。
%    \begin{macrocode}
  \ifnum \c@tocdepth >\@ne
%    \end{macrocode}
% Add some vertical space.
%    \begin{macrocode}
    \vskip \cftbeforesubsecskip
%    \end{macrocode}
% 创建一个组(group),使段落更改(paragraphing changes)保持为局部(local)。
% 将 |\leftskip| 设置为条目的缩进(entry's indentation)。
%    \begin{macrocode}
    {\leftskip \cftsubsecindent\relax
%    \end{macrocode}
% 将 |\rightskip| 设置为 |\@tocrmarg|,为页码留出空间。
%    \begin{macrocode}
     \rightskip \@tocrmarg
%    \end{macrocode}
% 确保填写条目的最后一行。如果将 |\parfillskip| 设置为负数,
% 则可以防止任何溢出盒子的消息(overfull box messages)。
%    \begin{macrocode}
     \parfillskip -\rightskip
%    \end{macrocode}
% 将段落缩进(paragraph indent)设置为条目的缩进(entry's indentation)。
%    \begin{macrocode}
     \parindent \cftsubsecindent\relax\@afterindenttrue
%    \end{macrocode}
% 尽量避免多行条目(multiple line entry)中的行之间出现中断(breaks)。
%    \begin{macrocode}
     \interlinepenalty\@M
%    \end{macrocode}
% 确保我们已经是左垂直模式(left vertical mode)。
%    \begin{macrocode}
     \leavevmode
%    \end{macrocode}
% 我们的 |\numberline| 版本要求数字盒子(number box)的宽度以 |\@tempdima| 为单位,
% 并定义了三个宏 |\@cftbsnum|、|\@cftasnum| 和 |\@cftasnumb|。我们将所有这些设置为该条目的值。
%    \begin{macrocode}
     \@tempdima \cftsubsecnumwidth\relax
     \let\@cftbsnum \cftsubsecpresnum
     \let\@cftasnum \cftsubsecaftersnum
     \let\@cftasnumb \cftsubsecaftersnumb
%    \end{macrocode}
% 安排在当前缩进处(current indent)设置标题的(节编号和)第一行,并进一步缩进任何其他行。
%    \begin{macrocode}
     \advance\leftskip \@tempdima \null\nobreak\hskip -\leftskip
%    \end{macrocode}
% 打印(编号和)标题,禁止任何断行(breaking)。
%    \begin{macrocode}
     {\cftsubsecfont #1}\nobreak
%    \end{macrocode}
% 打印引线(leader)和页码,然后关闭组(group)。
%    \begin{macrocode}
     \cftsubsecfillnum{#2}}%
  \fi}
%    \end{macrocode}
% \end{macro}
%
% \begin{macro}{\cftbeforesubsecskip}
% \begin{macro}{\cftsubsecindent}
% \begin{macro}{\cftsubsecnumwidth}
% \begin{macro}{\cftsubsecfont}
% \begin{macro}{\cftsubsecpresnum}
% \begin{macro}{\cftsubsecaftersnum}
% \begin{macro}{\cftsubsecaftersnumb}
% \begin{macro}{\cftsubsecleader}
% \begin{macro}{\cftsubsecdotsep}
% \begin{macro}{\cftsubsecpagefont}
% \begin{macro}{\cftsubsecafterpnum}
% 这些是控制子节条目(Sub-section entries)排版的用户命令。
%  它们经过了初始化,以提供标准外观(standard appearance)。
%    \begin{macrocode}
\newlength{\cftbeforesubsecskip}
  \setlength{\cftbeforesubsecskip}{\z@ \@plus.2\p@}
\newlength{\cftsubsecindent}
\newlength{\cftsubsecnumwidth}
\if@cfthaschapter
  \setlength{\cftsubsecindent}{3.8em}
  \setlength{\cftsubsecnumwidth}{3.2em}
\else
  \setlength{\cftsubsecindent}{1.5em}
  \setlength{\cftsubsecnumwidth}{2.3em}
\fi
\newcommand{\cftsubsecfont}{\normalfont}
\newcommand{\cftsubsecpresnum}{}
\newcommand{\cftsubsecaftersnum}{}
\newcommand{\cftsubsecaftersnumb}{}
\newcommand{\cftsubsecleader}{\normalfont\cftdotfill{\cftsubsecdotsep}}
\newcommand{\cftsubsecdotsep}{\cftdotsep}
\newcommand{\cftsubsecpagefont}{\normalfont}
\newcommand{\cftsubsecafterpnum}{}
%    \end{macrocode}
% \end{macro}
% \end{macro}
% \end{macro}
% \end{macro}
% \end{macro}
% \end{macro}
% \end{macro}
% \end{macro}
% \end{macro}
% \end{macro}
% \end{macro}
%
% \begin{macro}{\cftsubsecfillnum}
% |\cftsubsecfillnum{|\meta{page}|}| 排版小节条目(|subsection| entry)的引线(leader)和页码(\meta{page} number)。
% 首先打印引线(leader),然后,没有断行(no break),在宽度为 |\@pnumwidth| 的盒子中
% 设置页码(page number),不要忘了结束段落(finish the paragraph)。
%    \begin{macrocode}
\newcommand{\cftsubsecfillnum}[1]{%
  {\cftsubsecleader}\nobreak
  \makebox[\@pnumwidth][\cftpnumalign]{
    \cftsubsecpagefont #1}\cftsubsecafterpnum\par
}
%    \end{macrocode}
% \end{macro}
%
% \begin{macro}{\l@subsubsection}
% |\l@subsubsection{|\meta{title}|}{|\meta{page}|}| 排版子子节标题(|subsubsection| heading)的 ToC 条目。
% 它是默认 |\l@subsubsection| 的参数化副本(parameterised copy)(请参见 \file{classes.dtx}\ 以获取原始定义)。
%    \begin{macrocode}
\renewcommand*{\l@subsubsection}[2]{%
  \ifnum \c@tocdepth >\tw@
    \vskip \cftbeforesubsubsecskip
    {\leftskip \cftsubsubsecindent\relax
     \rightskip \@tocrmarg
     \parfillskip -\rightskip
     \parindent \cftsubsubsecindent\relax\@afterindenttrue
     \interlinepenalty\@M
     \leavevmode
     \@tempdima \cftsubsubsecnumwidth\relax
     \let\@cftbsnum \cftsubsubsecpresnum
     \let\@cftasnum \cftsubsubsecaftersnum
     \let\@cftasnumb \cftsubsubsecaftersnumb
     \advance\leftskip \@tempdima \null\nobreak\hskip -\leftskip
     {\cftsubsubsecfont #1}\nobreak
     \cftsubsubsecfillnum{#2}}%
  \fi}
%    \end{macrocode}
% \end{macro}
%
% \begin{macro}{\cftbeforesubsubsecskip}
% \begin{macro}{\cftsubsubsecindent}
% \begin{macro}{\cftsubsubsecnumwidth}
% \begin{macro}{\cftsubsubsecfont}
% \begin{macro}{\cftsubsubsecpresnum}
% \begin{macro}{\cftsubsubsecaftersnum}
% \begin{macro}{\cftsubsubsecaftersnumb}
% \begin{macro}{\cftsubsubsecleader}
% \begin{macro}{\cftsubsubsecdotsep}
% \begin{macro}{\cftsubsubsecpagefont}
% \begin{macro}{\cftsubsubsecafterpnum}
% \begin{macro}{\cftsubsubsecfillnum}
% 这些是控制子子节条目(Sub-sub-section entries)排版的用户命令。
%  它们经过了初始化,以提供标准外观(standard appearance)。
%    \begin{macrocode}
\newlength{\cftbeforesubsubsecskip}
  \setlength{\cftbeforesubsubsecskip}{\z@ \@plus.2\p@}
\newlength{\cftsubsubsecindent}
\newlength{\cftsubsubsecnumwidth}
\if@cfthaschapter
  \setlength{\cftsubsubsecindent}{7.0em}
  \setlength{\cftsubsubsecnumwidth}{4.1em}
\else
  \setlength{\cftsubsubsecindent}{3.8em}
  \setlength{\cftsubsubsecnumwidth}{3.2em}
\fi
\newcommand{\cftsubsubsecfont}{\normalfont}
\newcommand{\cftsubsubsecpresnum}{}
\newcommand{\cftsubsubsecaftersnum}{}
\newcommand{\cftsubsubsecaftersnumb}{}
\newcommand{\cftsubsubsecleader}{\normalfont\cftdotfill{\cftsubsubsecdotsep}}
\newcommand{\cftsubsubsecdotsep}{\cftdotsep}
\newcommand{\cftsubsubsecpagefont}{\normalfont}
\newcommand{\cftsubsubsecafterpnum}{}
\newcommand{\cftsubsubsecfillnum}[1]{%
  {\cftsubsubsecleader}\nobreak
  \makebox[\@pnumwidth][\cftpnumalign]{
    \cftsubsubsecpagefont #1}\cftsubsubsecafterpnum\par
}
%    \end{macrocode}
% \end{macro}
% \end{macro}
% \end{macro}
% \end{macro}
% \end{macro}
% \end{macro}
% \end{macro}
% \end{macro}
% \end{macro}
% \end{macro}
% \end{macro}
% \end{macro}
%
% \begin{macro}{\l@paragraph}
% |\l@paragraph{|\meta{title}|}{|\meta{page}|}| 排版段落标题(|paragraph| heading)的 ToC 条目。
% 它是默认 |\l@paragraph| 的参数化副本(parameterised copy)(请参见 \file{classes.dtx}\ 以获取原始定义)。
%    \begin{macrocode}
\renewcommand*{\l@paragraph}[2]{%
  \ifnum \c@tocdepth >3\relax
    \vskip \cftbeforeparaskip
    {\leftskip \cftparaindent\relax
     \rightskip \@tocrmarg
     \parfillskip -\rightskip
     \parindent \cftparaindent\relax\@afterindenttrue
     \interlinepenalty\@M
     \leavevmode
     \@tempdima \cftparanumwidth\relax
     \let\@cftbsnum \cftparapresnum
     \let\@cftasnum \cftparaaftersnum
     \let\@cftasnumb \cftparaaftersnumb
     \advance\leftskip \@tempdima \null\nobreak\hskip -\leftskip
     {\cftparafont #1}\nobreak
     \cftparafillnum{#2}}%
  \fi}
%    \end{macrocode}
% \end{macro}
%
% \begin{macro}{\cftbeforeparaskip}
% \begin{macro}{\cftparaindent}
% \begin{macro}{\cftparanumwidth}
% \begin{macro}{\cftparafont}
% \begin{macro}{\cftparapresnum}
% \begin{macro}{\cftparaaftersnum}
% \begin{macro}{\cftparaaftersnumb}
% \begin{macro}{\cftparaleader}
% \begin{macro}{\cftparadotsep}
% \begin{macro}{\cftparapagefont}
% \begin{macro}{\cftparaafterpnum}
% \begin{macro}{\cftparafillnum}
%  这些是控制段落条目(Paragraph entries)排版的用户命令。
%  它们经过了初始化,以提供标准外观(standard appearance)。
%    \begin{macrocode}
\newlength{\cftbeforeparaskip}
  \setlength{\cftbeforeparaskip}{\z@ \@plus.2\p@}
\newlength{\cftparaindent}
\newlength{\cftparanumwidth}
\if@cfthaschapter
  \setlength{\cftparaindent}{10em}
  \setlength{\cftparanumwidth}{5em}
\else
  \setlength{\cftparaindent}{7.0em}
  \setlength{\cftparanumwidth}{4.1em}
\fi
\newcommand{\cftparafont}{\normalfont}
\newcommand{\cftparapresnum}{}
\newcommand{\cftparaaftersnum}{}
\newcommand{\cftparaaftersnumb}{}
\newcommand{\cftparaleader}{\normalfont\cftdotfill{\cftparadotsep}}
\newcommand{\cftparadotsep}{\cftdotsep}
\newcommand{\cftparapagefont}{\normalfont}
\newcommand{\cftparaafterpnum}{}
\newcommand{\cftparafillnum}[1]{%
  {\cftparaleader}\nobreak
  \makebox[\@pnumwidth][\cftpnumalign]{
    \cftparapagefont #1}\cftparaafterpnum\par
}
%    \end{macrocode}
% \end{macro}
% \end{macro}
% \end{macro}
% \end{macro}
% \end{macro}
% \end{macro}
% \end{macro}
% \end{macro}
% \end{macro}
% \end{macro}
% \end{macro}
% \end{macro}
%
% \begin{macro}{\l@subparagraph}
% |\l@subparagraph{|\meta{title}|}{|\meta{page}|}| 为子段标题(|subparagraph| heading)排版 ToC 条目。
% 它是默认 |\l@subparagraph| 的参数化副本(parameterised copy)(请参见 \file{classes.dtx}\ 以获取原始定义)。
%    \begin{macrocode}
\renewcommand*{\l@subparagraph}[2]{%
  \ifnum \c@tocdepth >4\relax
    \vskip \cftbeforesubparaskip
    {\leftskip \cftsubparaindent\relax
     \rightskip \@tocrmarg
     \parfillskip -\rightskip
     \parindent \cftsubparaindent\relax\@afterindenttrue
     \interlinepenalty\@M
     \leavevmode
     \@tempdima \cftsubparanumwidth\relax
     \let\@cftbsnum \cftsubparapresnum
     \let\@cftasnum \cftsubparaaftersnum
     \let\@cftasnumb \cftsubparaaftersnumb
     \advance\leftskip \@tempdima \null\nobreak\hskip -\leftskip
     {\cftsubparafont #1}\nobreak
     \cftsubparafillnum{#2}}%
  \fi}
%    \end{macrocode}
% \end{macro}
% \clearpage
% \begin{macro}{\cftbeforesubparaskip}
% \begin{macro}{\cftsubparaindent}
% \begin{macro}{\cftsubparanumwidth}
% \begin{macro}{\cftsubparafont}
% \begin{macro}{\cftsubparapresnum}
% \begin{macro}{\cftsubparaaftersnum}
% \begin{macro}{\cftsubparaaftersnumb}
% \begin{macro}{\cftsubparaleader}
% \begin{macro}{\cftsubparadotsep}
% \begin{macro}{\cftsubparapagefont}
% \begin{macro}{\cftsubparaafterpnum}
% \begin{macro}{\cftsubparafillnum}
% 这些是控制子段落条目(Sub-paragraph entries)排版的用户命令。它们经过了初始化,
% 以提供标准外观(standard appearance)。
%    \begin{macrocode}
\newlength{\cftbeforesubparaskip}
  \setlength{\cftbeforesubparaskip}{\z@ \@plus.2\p@}
\newlength{\cftsubparaindent}
\newlength{\cftsubparanumwidth}
\if@cfthaschapter
  \setlength{\cftsubparaindent}{12em}
  \setlength{\cftsubparanumwidth}{6em}
\else
  \setlength{\cftsubparaindent}{10em}
  \setlength{\cftsubparanumwidth}{5em}
\fi
\newcommand{\cftsubparafont}{\normalfont}
\newcommand{\cftsubparapresnum}{}
\newcommand{\cftsubparaaftersnum}{}
\newcommand{\cftsubparaaftersnumb}{}
\newcommand{\cftsubparaleader}{\normalfont\cftdotfill{\cftsubparadotsep}}
\newcommand{\cftsubparadotsep}{\cftdotsep}
\newcommand{\cftsubparapagefont}{\normalfont}
\newcommand{\cftsubparaafterpnum}{}
\newcommand{\cftsubparafillnum}[1]{%
  {\cftsubparaleader}\nobreak
  \makebox[\@pnumwidth][\cftpnumalign]{
    \cftsubparapagefont #1}\cftsubparaafterpnum\par
}
%    \end{macrocode}
% \end{macro}
% \end{macro}
% \end{macro}
% \end{macro}
% \end{macro}
% \end{macro}
% \end{macro}
% \end{macro}
% \end{macro}
% \end{macro}
% \end{macro}
% \end{macro}
%
%
% \begin{macro}{\@cftdobiblof}
% 如果使用了 \Lpack{tocbibind}\ 宏包,并且它重新定义了 |\listoffigures|,
% 我们需要满足这一点(cater for that)。定义的内容(contents of the definition)在 \Lpack{tocbibind}\ 中定义。
%    \begin{macrocode}
\newcommand{\@cftdobiblof}{%
  \if@dotoclof
    \if@bibchapter
      \phantomsection
      \addcontentsline{toc}{chapter}{\listfigurename}
    \else
      \phantomsection
      \addcontentsline{toc}{\@tocextra}{\listfigurename}
    \fi
  \fi}

%    \end{macrocode}
% \end{macro}
%
%
% \begin{macro}{\listoffigures}
% 这是默认 |\listoffigures| 命令的参数化版本(parameterised version)。
% 更改(changes)将推迟到加载所有的宏包之后。每个类(class)都有自己的定义,
% 但我们必须在一个定义中满足所有类,因此需要进行一些检查。
% 首先,对多栏(multicolumns)执行默认检查。(如果使用 \Lopt{titles}\ 选项,则不执行任何操作)。
%    \begin{macrocode}
\AtBeginDocument{%
\if@cftnctoc\else
\renewcommand{\listoffigures}{%
  \@cfttocstart
%    \end{macrocode}
%确保前面的段落(previous paragraph)已经结束(finished)。在组内(Within a group)设置
%局部段落样式(local paragraphing style)。排版标题,然后执行 \file{.lof}\ 文件的内容。
%    \begin{macrocode}
  \par
  \begingroup
    \parindent\z@ \parskip\cftparskip
    \@cftmakeloftitle
    \if@cfttocbibind
      \@cftdobiblof
    \fi
    \@starttoc{lof}%
  \endgroup
%    \end{macrocode}
% Finally, restore any multicolumn typesetting.
%    \begin{macrocode}
  \@cfttocfinish}%
\fi
}

%    \end{macrocode}
% \end{macro}
%
% \begin{macro}{\@cftmakeloftitle}
% 此命令排版图形目录(LoF)的标题(title)。
% \changes{v2.3}{2002/06/15}{Added \cs{@secpenalty} to \cs{cftmakeloftitle}}
% \changes{v2.3}{2002/06/15}{Added \cs{@cftpagestyle} to \cs{cftmakeloftitle}}
%    \begin{macrocode}
\newcommand{\@cftmakeloftitle}{%
  \addpenalty\@secpenalty
  \if@cfthaschapter
    \vspace*{\cftbeforeloftitleskip}
  \else
    \vspace{\cftbeforeloftitleskip}
  \fi
  \@cftpagestyle
  {\interlinepenalty\@M
  {\cftloftitlefont\listfigurename}{\cftafterloftitle}
  \cftmarklof
  \par\nobreak
  \vskip \cftafterloftitleskip
  \@afterheading}}

%    \end{macrocode}
% \end{macro}
%
% \begin{macro}{\cftbeforeloftitleskip}
% \begin{macro}{\cftafterloftitleskip}
%  这两个长度控制图形目录(LoF)标题前后的垂直间距(vertical spacing)。
%    \begin{macrocode}
\newlength{\cftbeforeloftitleskip}
\newlength{\cftafterloftitleskip}
%    \end{macrocode}
% 它们的值取决于文档是否有章(chapters)。在分章文档(chaptered documents)中,默认的
% 图形目录(LoF)标题被排版为 |\chapter*|,而非分章文档将图形目录(LoF)标题被排版为 |\section*|。
%    \begin{macrocode}
\if@cfthaschapter
  \setlength{\cftbeforeloftitleskip}{50pt}
  \setlength{\cftafterloftitleskip}{40pt}
\else
  \setlength{\cftbeforeloftitleskip}{3.5ex \@plus 1ex \@minus .2ex}
  \setlength{\cftafterloftitleskip}{2.3ex \@plus.2ex}
\fi
%    \end{macrocode}
% \end{macro}
% \end{macro}
%
% \begin{macro}{\cftloftitlefont}
% \begin{macro}{\cftafterloftitle}
% 图形目录(LoF)标题按 |\cftloftitlefont| 给出的样式排版。宏 |\cftafterloftitle| 在
% 排版标题后被调用。这被初始化为不执行任何操作。这两个宏都可以被重新定义以执行
% 其他操作(例如,在 |\cftloftitlefont| 中添加 |\hfill| 会使标题右对齐[flushright])。
% \changes{v2.3}{2002/06/15}{koma has different settings for LoF titles}
%    \begin{macrocode}
\if@cfthaschapter
  \newcommand{\cftloftitlefont}{\normalfont\Huge\bfseries}
  \if@cftkoma\renewcommand{\cftloftitlefont}{\size@chapter\sectfont}\fi
\else
  \newcommand{\cftloftitlefont}{\normalfont\Large\bfseries}
  \if@cftkoma\renewcommand{\cftloftitlefont}{\size@section\sectfont}\fi
\fi
\newcommand{\cftafterloftitle}{}

%    \end{macrocode}
% \end{macro}
% \end{macro}
%
% \begin{macro}{\l@figure}
% |\l@figure{|\meta{title}|}{|\meta{page}|}| 排版图形标题(|figure| caption heading)的 LoF 条目。
% 它是默认 |\l@figure| 的参数化副本(parameterised copy)(请参见 \file{classes.dtx}\ 以获取原始定义)。
%    \begin{macrocode}
\renewcommand*{\l@figure}[2]{%
  \ifnum \c@lofdepth >\z@
    \vskip \cftbeforefigskip
    {\leftskip \cftfigindent\relax
     \rightskip \@tocrmarg
     \parfillskip -\rightskip
     \parindent \cftfigindent\relax\@afterindenttrue
     \interlinepenalty\@M
     \leavevmode
     \@tempdima \cftfignumwidth\relax
     \let\@cftbsnum \cftfigpresnum
     \let\@cftasnum \cftfigaftersnum
     \let\@cftasnumb \cftfigaftersnumb
     \advance\leftskip \@tempdima \null\nobreak\hskip -\leftskip
     {\cftfigfont #1}\nobreak
     \cftfigfillnum{#2}}%
   \fi
  }
%    \end{macrocode}
% \end{macro}
%
% \begin{macro}{\cftbeforefigskip}
% \begin{macro}{\cftfigindent}
% \begin{macro}{\cftfignumwidth}
% \begin{macro}{\cftfigfont}
% \begin{macro}{\cftfigpresnum}
% \begin{macro}{\cftfigaftersnum}
% \begin{macro}{\cftfigaftersnumb}
% \begin{macro}{\cftfigleader}
% \begin{macro}{\cftfigdotsep}
% \begin{macro}{\cftfigpagefont}
% \begin{macro}{\cftfigafterpnum}
% \begin{macro}{\cftfigfillnum}
%  这些是用户命令,用于控制图标题条目(Figure caption entries)的排版。
%  它们经过了初始化,以提供标准外观(standard appearance)。
%    \begin{macrocode}
\newlength{\cftbeforefigskip}
  \setlength{\cftbeforefigskip}{\z@ \@plus.2\p@}
\newlength{\cftfigindent}
  \setlength{\cftfigindent}{1.5em}
\newlength{\cftfignumwidth}
  \setlength{\cftfignumwidth}{2.3em}
\newcommand{\cftfigfont}{\normalfont}
\newcommand{\cftfigpresnum}{}
\newcommand{\cftfigaftersnum}{}
\newcommand{\cftfigaftersnumb}{}
\newcommand{\cftfigleader}{\normalfont\cftdotfill{\cftfigdotsep}}
\newcommand{\cftfigdotsep}{\cftdotsep}
\newcommand{\cftfigpagefont}{\normalfont}
\newcommand{\cftfigafterpnum}{}
\newcommand{\cftfigfillnum}[1]{%
  {\cftfigleader}\nobreak
  \makebox[\@pnumwidth][\cftpnumalign]{\cftfigpagefont #1}
  \cftfigafterpnum\par
}
%    \end{macrocode}
% \end{macro}
% \end{macro}
% \end{macro}
% \end{macro}
% \end{macro}
% \end{macro}
% \end{macro}
% \end{macro}
% \end{macro}
% \end{macro}
% \end{macro}
% \end{macro}
%
% \begin{macro}{lofdepth}
% \begin{macro}{lotdepth}
% 计数器 |lofdepth| 和 |lotdepth| 由 \Lpack{subfigure}\ 宏包定义。如果未使用该宏包,
% 请在此处定义它们。
%    \begin{macrocode}
\if@cftsubfigopt\else
  \newcounter{lofdepth}\setcounter{lofdepth}{1}
  \newcounter{lotdepth}\setcounter{lotdepth}{1}
\fi

%    \end{macrocode}
% \end{macro}
% \end{macro}
%
% \begin{macro}{\@cftdobiblot}
% 如果使用了 \Lpack{tocbibind}\ 宏包,并且它重新定义了 |\listoftables|,
% 我们需要满足这一点(cater for that)。定义的内容(contents of the definition)在 \Lpack{tocbibind}\ 中定义。
%    \begin{macrocode}
\newcommand{\@cftdobiblot}{%
  \if@dotoclot
    \if@bibchapter
      \phantomsection
      \addcontentsline{toc}{chapter}{\listtablename}
    \else
      \phantomsection
      \addcontentsline{toc}{\@tocextra}{\listtablename}
    \fi
  \fi}

%    \end{macrocode}
% \end{macro}
%
%
% \begin{macro}{\listoftables}
% 这是默认 |\listoftables| 命令的参数化版本(parameterised version)。
% 更改(changes)将推迟到加载所有宏包之后。每个类(class)都有自己的定义,
% 但我们必须在一个定义中满足所有类,因此需要进行一些检查(checks)。首先,
% 对多栏(multicolumns)执行默认检查(default checks)。(如果已使用 \Lopt{titles}\ 选项,
% 则不执行任何操作)。
%    \begin{macrocode}
\AtBeginDocument{%
\if@cftnctoc\else
\renewcommand{\listoftables}{%
  \@cfttocstart
%    \end{macrocode}
% 确保前面的段落(previous paragraph)已经结束。在组内(Within a group)设置
% 局部段落样式(local paragraphing style)。排版标题,然后执行 \file{.lot}\ 文件的内容。
%    \begin{macrocode}
  \par
  \begingroup
    \parindent\z@ \parskip\cftparskip
    \@cftmakelottitle
    \if@cfttocbibind
      \@cftdobiblot
    \fi
    \@starttoc{lot}%
  \endgroup
%    \end{macrocode}
% 最后,恢复多列排版(multicolumn typesetting)。
%    \begin{macrocode}
  \@cfttocfinish}%
\fi
}

%    \end{macrocode}
% \end{macro}
%
% \begin{macro}{\@cftmakelottitle}
% 此命令排版表格目录(LoT)的标题。
% \changes{v2.3}{2002/06/15}{Added \cs{@secpenalty} to \cs{@cftmakelottitle}}
% \changes{v2.3}{2002/06/15}{Added \cs{@cftpagestyle} to \cs{@cftmakelottitle}}
%    \begin{macrocode}
\newcommand{\@cftmakelottitle}{%
  \addpenalty\@secpenalty
  \if@cfthaschapter
    \vspace*{\cftbeforelottitleskip}
  \else
    \vspace{\cftbeforelottitleskip}
  \fi
  \@cftpagestyle
  {\interlinepenalty\@M
  {\cftlottitlefont\listtablename}{\cftafterlottitle}
  \cftmarklot
  \par\nobreak
  \vskip \cftafterlottitleskip
  \@afterheading}}

%    \end{macrocode}
% \end{macro}
%
% \begin{macro}{\cftbeforelottitleskip}
% \begin{macro}{\cftafterlottitleskip}
%  这两个长度控制表格目录(LoT)标题前后的垂直间距(vertical spacing)。
%    \begin{macrocode}
\newlength{\cftbeforelottitleskip}
\newlength{\cftafterlottitleskip}
%    \end{macrocode}
% 它们的值取决于文档是否有章(chapters)。在分节文档(chaptered documents)中,
% 默认的表格目录(LoT)标题被排版为 |\chapter*|,而在非分章文档中,
% 表格目录(LoT)标题被排版为 |\section*|。
%    \begin{macrocode}
\if@cfthaschapter
  \setlength{\cftbeforelottitleskip}{50pt}
  \setlength{\cftafterlottitleskip}{40pt}
\else
  \setlength{\cftbeforelottitleskip}{3.5ex \@plus 1ex \@minus .2ex}
  \setlength{\cftafterlottitleskip}{2.3ex \@plus.2ex}
\fi
%    \end{macrocode}
% \end{macro}
% \end{macro}
%
% \begin{macro}{\cftlottitlefont}
% \begin{macro}{\cftafterlottitle}
% 表格目录(LoT)标题按 |\cftlottitlefont| 给出的样式排版。宏 |\cftafterlottitle| 在
% 排版标题后被调用。这被初始化为不执行任何操作。这两个宏都可以被重新定义以执行
% 其他操作(例如,在 |\cftlottitlefont| 中添加 |\hfill| 会使标题右对齐)。
% \changes{v2.3}{2002/06/15}{koma has different settings for LoT titles}
%    \begin{macrocode}
\if@cfthaschapter
  \newcommand{\cftlottitlefont}{\normalfont\Huge\bfseries}
  \if@cftkoma\renewcommand{\cftlottitlefont}{\size@chapter\sectfont}\fi
\else
  \newcommand{\cftlottitlefont}{\normalfont\Large\bfseries}
  \if@cftkoma\renewcommand{\cftlottitlefont}{\size@section\sectfont}\fi
\fi
\newcommand{\cftafterlottitle}{}

%    \end{macrocode}
% \end{macro}
% \end{macro}
%
% \begin{macro}{\l@table}
%  |\l@table{|\meta{title}|}{|\meta{page}|}| 为表格标题(|table| caption heading)排
%  版 LoT 条目。它是默认 |\l@table| 的参数化副本(parameterised copy)(有关原始定义,
%  请参见 \file{classes.dtx}\ 以获取原始定义)。
%    \begin{macrocode}
\renewcommand*{\l@table}[2]{%
  \ifnum\c@lotdepth >\z@
    \vskip \cftbeforetabskip
    {\leftskip \cfttabindent\relax
     \rightskip \@tocrmarg
     \parfillskip -\rightskip
     \parindent \cfttabindent\relax\@afterindenttrue
     \interlinepenalty\@M
     \leavevmode
     \@tempdima \cfttabnumwidth\relax
     \let\@cftbsnum \cfttabpresnum
     \let\@cftasnum \cfttabaftersnum
     \let\@cftasnumb \cfttabaftersnumb
     \advance\leftskip \@tempdima \null\nobreak\hskip -\leftskip
     {\cfttabfont #1}\nobreak
     \cfttabfillnum{#2}}%
   \fi
  }
%    \end{macrocode}
% \end{macro}
%
% \begin{macro}{\cftbeforetabskip}
% \begin{macro}{\cfttabindent}
% \begin{macro}{\cfttabnumwidth}
% \begin{macro}{\cfttabfont}
% \begin{macro}{\cfttabpresnum}
% \begin{macro}{\cfttabaftersnum}
% \begin{macro}{\cfttabaftersnumb}
% \begin{macro}{\cfttableader}
% \begin{macro}{\cfttabdotsep}
% \begin{macro}{\cfttabpagefont}
% \begin{macro}{\cfttabafterpnum}
% \begin{macro}{\cfttabfillnum}
%  这些是用户命令,用于控制表格标题条目(Table caption entries)的排版。
%  它们经过了初始化,以提供标准外观(standard appearance)。
%    \begin{macrocode}
\newlength{\cftbeforetabskip}
  \setlength{\cftbeforetabskip}{\z@ \@plus.2\p@}
\newlength{\cfttabindent}
  \setlength{\cfttabindent}{1.5em}
\newlength{\cfttabnumwidth}
  \setlength{\cfttabnumwidth}{2.3em}
\newcommand{\cfttabfont}{\normalfont}
\newcommand{\cfttabpresnum}{}
\newcommand{\cfttabaftersnum}{}
\newcommand{\cfttabaftersnumb}{}
\newcommand{\cfttableader}{\normalfont\cftdotfill{\cfttabdotsep}}
\newcommand{\cfttabdotsep}{\cftdotsep}
\newcommand{\cfttabpagefont}{\normalfont}
\newcommand{\cfttabafterpnum}{}
\newcommand{\cfttabfillnum}[1]{%
  {\cfttableader}\nobreak
  \makebox[\@pnumwidth][\cftpnumalign]{\cfttabpagefont #1}\cfttabafterpnum\par
}
%    \end{macrocode}
% \end{macro}
% \end{macro}
% \end{macro}
% \end{macro}
% \end{macro}
% \end{macro}
% \end{macro}
% \end{macro}
% \end{macro}
% \end{macro}
% \end{macro}
% \end{macro}
%
% \subsection[对 \Lpack{subfigure}\ 宏包的支持]{{\heiti 对} \Lpack{subfigure}\ {\heiti 宏包的支持}}
% \changes{v1.1}{2000/02/12}{Added subfigure support}
%
% 支持 \Lpack{subfigure}\ 宏包的代码在本质上与图形(figure)和表格(table)标题的代码相同,
% 只有名称被更改。但是,只有在 \Lpack{subfigure}\ 宏包实际加载时才需要执行代码。
%
% \begin{macro}{\@cftl@subfig}
%    此命令重新定义了 |\l@subfigure| 命令。
%    \begin{macrocode}
\newcommand{\@cftl@subfig}{%
%    \end{macrocode}
% \begin{macro}{\l@subfigure}
%  |\l@subfigure{|\meta{title}|}{|\meta{page}|}| 为子图形标题(|subfigure| caption heading)排版 LoF 条目。
%  它本质上与 |\l@figure| 的参数化代码(parameterised code)相同,除了必须考虑到 |lofdepth|。
%    \begin{macrocode}
\renewcommand*{\l@subfigure}[2]{%
  \ifnum \c@lofdepth > \toclevel@subfigure
    \vskip \cftbeforesubfigskip
    {\leftskip \cftsubfigindent\relax
     \rightskip \@tocrmarg
     \parfillskip -\rightskip
     \parindent \cftsubfigindent\relax\@afterindenttrue
     \interlinepenalty\@M
     \leavevmode
     \@tempdima \cftsubfignumwidth\relax
     \let\@cftbsnum \cftsubfigpresnum
     \let\@cftasnum \cftsubfigaftersnum
     \let\@cftasnumb \cftsubfigaftersnumb
     \advance\leftskip \@tempdima \null\nobreak\hskip -\leftskip
     {\cftsubfigfont ##1}\nobreak
     \cftsubfigfillnum{##2}}%
  \fi
  }%
}

%    \end{macrocode}
% \end{macro}
% \end{macro}
%
% \begin{macro}{\@cftsetsubfig}
% 此命令初始化 LoF 中的子图形标题(subfigure captions)的设置。
%    \begin{macrocode}
\newcommand{\@cftsetsubfig}{%
%    \end{macrocode}
% \begin{macro}{\cftbeforesubfigskip}
% \begin{macro}{\cftsubfigindent}
% \begin{macro}{\cftsubfignumwidth}
% \begin{macro}{\cftsubfigfont}
% \begin{macro}{\cftsubfigpresnum}
% \begin{macro}{\cftsubfigaftersnum}
% \begin{macro}{\cftsubfigaftersnumb}
% \begin{macro}{\cftsubfigleader}
% \begin{macro}{\cftsubfigdotsep}
% \begin{macro}{\cftsubfigpagefont}
% \begin{macro}{\cftsubfigafterpnum}
% \begin{macro}{\toclevel@subfig}
% \begin{macro}{\cftsubfigfillnum}
%    \begin{macrocode}
\newlength{\cftbeforesubfigskip}
  \setlength{\cftbeforesubfigskip}{\z@ \@plus.2\p@}
\newlength{\cftsubfigindent}
  \setlength{\cftsubfigindent}{3.8em}
\newlength{\cftsubfignumwidth}
  \setlength{\cftsubfignumwidth}{2.5em}
\newcommand{\cftsubfigfont}{\normalfont}
\newcommand{\cftsubfigpresnum}{}
\newcommand{\cftsubfigaftersnum}{}
\newcommand{\cftsubfigaftersnumb}{}
\newcommand{\cftsubfigleader}{\normalfont\cftdotfill{\cftsubtabdotsep}}
\newcommand{\cftsubfigdotsep}{\cftdotsep}
\newcommand{\cftsubfigpagefont}{\normalfont}
\newcommand{\cftsubfigafterpnum}{}
\providecommand{\toclevel@subfigure}{1}
\newcommand{\cftsubfigfillnum}[1]{%
  {\cftsubfigleader}\nobreak
  \makebox[\@pnumwidth][\cftpnumalign]{
    \cftsubfigpagefont ##1}\cftsubfigafterpnum\par
}
%    \end{macrocode}
% \end{macro}
% \end{macro}
% \end{macro}
% \end{macro}
% \end{macro}
% \end{macro}
% \end{macro}
% \end{macro}
% \end{macro}
% \end{macro}
% \end{macro}
% \end{macro}
% \end{macro}
% 这是 |\@cftsetsubfig| 的结尾(end)。
%    \begin{macrocode}
}

%    \end{macrocode}
% \end{macro}
%
% \begin{macro}{\@cftl@subtab}
%    此代码重新定义了 |\l@subtable| 的代码。
%    \begin{macrocode}
\newcommand{\@cftl@subtab}{%
%    \end{macrocode}
% \begin{macro}{\l@subtable}
%  |\l@subtable{|\meta{title}|}{|\meta{page}|}| 为子表格标题(|subtable| caption heading)设置 LoT 条目。
%  它本质上与 |\l@table| 的参数化代码(parameterised code)相同,除了必须考虑到 |lotdepth|。
%    \begin{macrocode}
\renewcommand*{\l@subtable}[2]{%
  \ifnum \c@lotdepth > \toclevel@subtable
    \vskip \cftbeforesubtabskip
    {\leftskip \cftsubtabindent\relax
     \rightskip \@tocrmarg
     \parfillskip -\rightskip
     \parindent \cftsubtabindent\relax\@afterindenttrue
     \interlinepenalty\@M
     \leavevmode
     \@tempdima \cftsubtabnumwidth\relax
     \let\@cftbsnum \cftsubtabpresnum
     \let\@cftasnum \cftsubtabaftersnum
     \let\@cftasnumb \cftsubtabaftersnumb
     \advance\leftskip \@tempdima \null\nobreak\hskip -\leftskip
     {\cftsubtabfont ##1}\nobreak
     \cftsubtabfillnum{##2}}%
  \fi
  }%
}
%    \end{macrocode}
% \end{macro}
% \end{macro}
%
% \begin{macro}{\@cftsetsubtab}
% 此命令为 LoT 中的子表格条目(subtable entries)设置默认值。
%    \begin{macrocode}
\newcommand{\@cftsetsubtab}{%
%    \end{macrocode}
% \begin{macro}{\cftbeforesubtabskip}
% \begin{macro}{\cftsubtabindent}
% \begin{macro}{\cftsubtabnumwidth}
% \begin{macro}{\cftsubtabfont}
% \begin{macro}{\cftsubtabpresnum}
% \begin{macro}{\cftsubtabaftersnum}
% \begin{macro}{\cftsubtabaftersnumb}
% \begin{macro}{\cftsubtableader}
% \begin{macro}{\cftsubtabdotsep}
% \begin{macro}{\cftsubtabpagefont}
% \begin{macro}{\cftsubtabafterpnum}
% \begin{macro}{\toclevel@subtable}
% \begin{macro}{\cftsubtabfillnum}
%  这些是用于控制子格表标题条目(Subtable caption entries)排版的用户命令。它们经过了初始化,
%  以提供标准外观(standard appearance)。
%    \begin{macrocode}
\newlength{\cftbeforesubtabskip}
  \setlength{\cftbeforesubtabskip}{\z@ \@plus.2\p@}
\newlength{\cftsubtabindent}
  \setlength{\cftsubtabindent}{3.8em}
\newlength{\cftsubtabnumwidth}
  \setlength{\cftsubtabnumwidth}{2.5em}
\newcommand{\cftsubtabfont}{\normalfont}
\newcommand{\cftsubtabpresnum}{}
\newcommand{\cftsubtabaftersnum}{}
\newcommand{\cftsubtabaftersnumb}{}
\newcommand{\cftsubtableader}{\normalfont\cftdotfill{\cftsubtabdotsep}}
\newcommand{\cftsubtabdotsep}{\cftdotsep}
\newcommand{\cftsubtabpagefont}{\normalfont}
\newcommand{\cftsubtabafterpnum}{}
\providecommand{\toclevel@subtable}{1}
\newcommand{\cftsubtabfillnum}[1]{%
  {\cftsubtableader}\nobreak
  \makebox[\@pnumwidth][\cftpnumalign]{\cftsubtabpagefont ##1}
  \cftsubtabafterpnum\par
}
%    \end{macrocode}
% \end{macro}
% \end{macro}
% \end{macro}
% \end{macro}
% \end{macro}
% \end{macro}
% \end{macro}
% \end{macro}
% \end{macro}
% \end{macro}
% \end{macro}
% \end{macro}
% \end{macro}
% 这是 |\@cftsetsubtab| 的结尾(end)。
%    \begin{macrocode}
}

%    \end{macrocode}
% \end{macro}
%
% 仅当指定了 \Lopt{subfigure}\ 选项时,才调用 \Lpack{subfigure}\ 宏包设置代码。
% |\l@...| 的重定义(redefinitions)必须在加载了 \Lpack{subfigure}\ 宏包之后进行。
%    \begin{macrocode}

\if@cftsubfigopt
  \@cftsetsubfig\@cftsetsubtab
  \AtBeginDocument{\@cftl@subfig\@cftl@subtab}
\fi
%%  \AtBeginDocument{\if@cftsubfigopt
%%    \@cftsetsubfig\@cftsetsubtab
%%    \@cftl@subfig\@cftl@subtab
%%  \fi}

%    \end{macrocode}
%
%
% \subsection[新列表\ \ldots]{{\heiti 新列表}\ \ldots}
% \changes{v2.0}{2001/03/03}{Added \cs{newlistof} and \cs{newlistentry}}
%
% \begin{macro}{\newlistentry}
% |\newlistentry[|\meta{within}|]{|\meta{counter}|}{|\meta{ext}|}{|\meta{level-1}|}| 为列表(List)中的新类型条目(new kind of entry)创建一组命令。
%    \begin{macrocode}
\newcommand{\newlistentry}[4][\@empty]{%
%    \end{macrocode}
% \begin{macro}{\c@X}
% \begin{macro}{\theX}
% 检查是否已定义 \meta{within}\ 和 \meta{counter}。如果没有定义 \meta{within},则出现错误。
% 如果已定义 \meta{counter},也会出现错误。设置默认计数器值(default counter values)。
%    \begin{macrocode}
  \@ifundefined{c@#2}{%    check & set the counter
    \ifx \@empty#1\relax
      \newcounter{#2}
    \else
      \@ifundefined{c@#1}{\PackageWarning{tocloft}%
                          {#1 has no counter for use as a `within'}
        \newcounter{#2}}%
      {\newcounter{#2}[#1]%
       \expandafter\edef\csname the#2\endcsname{%
         \expandafter\noexpand\csname the#1\endcsname.\noexpand\arabic{#2}}}
    \fi
    \setcounter{#2}{0}
  }
  {\PackageError{tocloft}{#2 has been previously defined}{\@eha}}

%    \end{macrocode}
% \end{macro}
% \end{macro}
%
% 这样就完成了错误检查(error checking)。无论结果如何,新命令的其余部分都已定义。
%
% \begin{macro}{\l@X}
% |\l@X{|\meta{title}|}{|\meta{page}|}| 排版条目。
%    \begin{macrocode}
  \@namedef{l@#2}##1##2{%
%    \end{macrocode}
% 只有当 |\Zdepth| 深度大于 \meta{level-1}\ 时才会排版。
%    \begin{macrocode}
    \ifnum \@nameuse{c@#3depth} > #4\relax
      \vskip \@nameuse{cftbefore#2skip}
      {\leftskip \@nameuse{cft#2indent}\relax
       \rightskip \@tocrmarg
       \parfillskip -\rightskip
       \parindent \@nameuse{cft#2indent}\relax\@afterindenttrue
       \interlinepenalty\@M
       \leavevmode
       \@tempdima \@nameuse{cft#2numwidth}\relax
       \expandafter\let\expandafter\@cftbsnum\csname cft#2presnum\endcsname
       \expandafter\let\expandafter\@cftasnum\csname cft#2aftersnum\endcsname
       \expandafter\let\expandafter\@cftasnumb\csname cft#2aftersnumb\endcsname
       \advance\leftskip\@tempdima \null\nobreak\hskip -\leftskip
       {\@nameuse{cft#2font}##1}\nobreak
       \@nameuse{cft#2fillnum}{##2}}%
    \fi
  }  % end of \l@#2

%    \end{macrocode}
% \end{macro}
%
% 现在定义 |\l@X| 使用的所有布局命令(layout commands)。它们的默认值对应于
% 非分章文档(non-chaptered documents)中节条目(section entries)的默认值。
% \begin{macro}{\cftbeforeXskip}
%    \begin{macrocode}
  \expandafter\newlength\csname cftbefore#2skip\endcsname
    \setlength{\@nameuse{cftbefore#2skip}}{\z@ \@plus .2\p@}
%    \end{macrocode}
% \end{macro}
% \begin{macro}{\cftXindent}
% \begin{macro}{\cftXnumwidth}
%    \begin{macrocode}
  \expandafter\newlength\csname cft#2indent\endcsname
  \expandafter\newlength\csname cft#2numwidth\endcsname
%    \end{macrocode}
% 根据条目的级别(level)设置缩进(indent)和编号宽度(numwidth)的默认值。级别 1 对应于图形条目(figure entry)。
%    \begin{macrocode}
  \ifcase #4\relax  % 0
    \setlength{\@nameuse{cft#2indent}}{0em}
    \setlength{\@nameuse{cft#2numwidth}}{1.5em}
  \or               % 1
    \setlength{\@nameuse{cft#2indent}}{1.5em}
    \setlength{\@nameuse{cft#2numwidth}}{2.3em}
  \or               % 2
    \setlength{\@nameuse{cft#2indent}}{3.8em}
    \setlength{\@nameuse{cft#2numwidth}}{3.2em}
  \or               % 3
    \setlength{\@nameuse{cft#2indent}}{7.0em}
    \setlength{\@nameuse{cft#2numwidth}}{4.1em}
  \else             % anything else
    \setlength{\@nameuse{cft#2indent}}{10.0em}
    \setlength{\@nameuse{cft#2numwidth}}{5.0em}
  \fi
%    \end{macrocode}
% \end{macro}
% \end{macro}
% \begin{macro}{\cftXfont}
% \begin{macro}{\cftXpresnum}
% \begin{macro}{\cftXaftersnum}
% \begin{macro}{\cftXaftersnumb}
% \begin{macro}{\cftXdotsep}
% \begin{macro}{\cftXleader}
% \begin{macro}{\cftXpagefont}
% \begin{macro}{\cftXafterpnum}
% And the remaining commands.
%    \begin{macrocode}
  \@namedef{cft#2font}{\normalfont}
  \@namedef{cft#2presnum}{}
  \@namedef{cft#2aftersnum}{}
  \@namedef{cft#2aftersnumb}{}
  \@namedef{cft#2dotsep}{\cftdotsep}
  \@namedef{cft#2leader}{\normalfont\cftdotfill{\@nameuse{cft#2dotsep}}}
  \@namedef{cft#2pagefont}{\normalfont}
  \@namedef{cft#2afterpnum}{}
%    \end{macrocode}
% \end{macro}
% \end{macro}
% \end{macro}
% \end{macro}
% \end{macro}
% \end{macro}
% \end{macro}
% \end{macro}
%
% \begin{macro}{\toclevel@X}
% Hyperref 宏包需要一个命令 |\toclevel@X|,以保存 \meta{level-1}\ 的值。
%    \begin{macrocode}
  \@namedef{toclevel@#2}{#4}
%    \end{macrocode}
% \end{macro}
%
% \begin{macro}{\cftXfillnum}
% 排版页眉(leader)和页码(page number)。
%    \begin{macrocode}
  \@namedef{cft#2fillnum}##1{%
    {\@nameuse{cft#2leader}}\nobreak
    \makebox[\@pnumwidth][\cftpnumalign]{\@nameuse{cft#2pagefont}##1}
    \@nameuse{cft#2afterpnum}\par
  }
%    \end{macrocode}
% \end{macro}
% 这样就结束了 |\newlistentry| 的定义。
%    \begin{macrocode}
}
%    \end{macrocode}
% \end{macro}
%
% \begin{macro}{\newlistof}
% |\newlistof[|\meta{within}|]{|\meta{entry}|}{|\meta{ext}|}{|\meta{listofname}|}| 创建新 List 的命令。
%    \begin{macrocode}
\newcommand{\newlistof}[4][\@empty]{%
%    \end{macrocode}
% 调用 |\newlistentry| 设置第一级条目(first level entry)。
%    \begin{macrocode}
  \ifx \@empty#1\relax
    \newlistentry{#2}{#3}{0}
  \else
    \newlistentry[#1]{#2}{#3}{0}
  \fi
%    \end{macrocode}
% \end{macro}
%
% \begin{macro}{\ext@Z}
% \begin{macro}{\Zdepth}
% 文件扩展名(file extension)和列表深度(listing depth)。
%    \begin{macrocode}
  \@namedef{ext@#2}{#3}
  \newcounter{#3depth}
  \setcounter{#3depth}{1}
%    \end{macrocode}
% \end{macro}
% \end{macro}
%
% \begin{macro}{\cftmarkZ}
% 列表(listing)的标题标记(heading marks)。
% \changes{v2.3}{2002/06/15}{different koma settings in \cs{newlistof}}
%    \begin{macrocode}
  \if@cftkoma
    \@namedef{cftmark#3}{%
      \@mkboth{#4}{#4}}
  \else
    \@namedef{cftmark#3}{%
      \@mkboth{\MakeUppercase{#4}}{\MakeUppercase{#4}}}
  \fi
%    \end{macrocode}
% \end{macro}
%
% \begin{macro}{\listofX}
% 排版列表标题(listing title)和条目(entries)。
%    \begin{macrocode}
 \if@cftnctoc
%    \end{macrocode}
% 对于 \Lopt{titles}\ 选项,基本上是从标准 |\tableofcontents| 命令复制代码。
%    \begin{macrocode}
  \@namedef{listof#2}{%
    \@cfttocstart
    \if@cfthaschapter
      \chapter*{#4}
    \else
      \section*{#4}
    \fi
    \@nameuse{cftmark#3}
    \@starttoc{#3}%
    \@cfttocfinish}
 \else
%    \end{macrocode}
% 否则,请使用完全参数化的定义(fully parameterised definition)。
%    \begin{macrocode}
  \@namedef{listof#2}{%
    \@cfttocstart
    \par
    \begingroup
      \parindent\z@ \parskip\cftparskip
      \@nameuse{@cftmake#3title}
      \@starttoc{#3}%
    \endgroup
    \@cfttocfinish}
 \fi

%    \end{macrocode}
% \end{macro}
%
% \begin{macro}{\@cftmakeZtitle}
% 排版标题。
% \changes{v2.3}{2002/06/15}{Added \cs{@secpenalty} to \cs{@cftmakeZtitle}}
% \changes{v2.3}{2002/06/15}{Added \cs{@cftpagestyle} to \cs{@cftmakeZtitle}}
%    \begin{macrocode}
  \@namedef{@cftmake#3title}{%
    \addpenalty\@secpenalty
    \if@cfthaschapter
      \vspace*{\@nameuse{cftbefore#3titleskip}}%
    \else
      \vspace{\@nameuse{cftbefore#3titleskip}}%
    \fi
    \@cftpagestyle
    {\interlinepenalty\@M
    {\@nameuse{cft#3titlefont}#4}{\@nameuse{cftafter#3title}}%
    \@nameuse{cftmark#3}%
    \par\nobreak
    \vskip \@nameuse{cftafter#3titleskip}%
    \@afterheading}}

%    \end{macrocode}
% \end{macro}
%
% \begin{macro}{\cftbeforeZtitleskip}
% \begin{macro}{\cftafterZtitleskip}
% \begin{macro}{\cftZtitlefont}
% 标题(title heading)之前和之后的间距(skips),以及标题字体(title font)。默认值取决于文档类(document class)是否具有章(chapters)。
%    \begin{macrocode}
   \expandafter\newlength\csname cftbefore#3titleskip\endcsname
   \expandafter\newlength\csname cftafter#3titleskip\endcsname
   \if@cfthaschapter
      \setlength{\@nameuse{cftbefore#3titleskip}}{50pt}
      \setlength{\@nameuse{cftafter#3titleskip}}{40pt}
      \if@cftkoma
        \@namedef{cft#3titlefont}{\size@chapter\sectfont}
      \else
        \@namedef{cft#3titlefont}{\normalfont\Huge\bfseries}
      \fi
    \else
      \setlength{\@nameuse{cftbefore#3titleskip}}{3.5ex \@plus 1ex \@minus .2ex}
      \setlength{\@nameuse{cftafter#3titleskip}}{2.3ex \@plus .2ex}
      \if@cftkoma
        \@namedef{cft#3titlefont}{\size@section\sectfont}
      \else
        \@namedef{cft#3titlefont}{\normalfont\Huge\bfseries}
      \fi
    \fi
%    \end{macrocode}
% \end{macro}
% \end{macro}
% \end{macro}
%
% \begin{macro}{\cftafterZtitle}
% 标题(title)后面的东西。
%    \begin{macrocode}
    \@namedef{cftafter#3title}{}
%    \end{macrocode}
% \end{macro}
%
% \begin{macro}{\cftZprehook}
% \begin{macro}{\cftZposthook}
% 条目列表(list of entries)前后的钩子(Hooks)。
%    \begin{macrocode}
    \@namedef{cft#3prehook}{}
    \@namedef{cft#3posthook}{}
%    \end{macrocode}
% \end{macro}
%
% 这是 |\newlistof| 定义的结尾(end)。
%    \begin{macrocode}
}
%    \end{macrocode}
% \end{macro}
%
% \begin{macro}{\cftsetindents}
% \changes{v2.0}{2001/03/15}{Added \cs{cftsetindents}}
% |\cftsetindents{|\meta{entry}|}{|\meta{indent}|}{|\meta{numwidth}|}| 设
% 置 \meta{entry}\ 条目 \meta{entry}\ 的{\kaiti 缩进}(\textit{indent})和{\kaiti 编号宽度}(\textit{numwidth})。
% 宏必须在外部条目名称(external entry name)和内部简写(internal shorthand)之间进行映射(map)。
%    \begin{macrocode}
\newcommand{\cftsetindents}[3]{%
  \def\@cftemp{#1}
  \ifx\@cftemp\cftchapname
    \@cftsetindents{chap}{#2}{#3}
  \else
    \ifx\@cftemp\cftsecname \@cftsetindents{sec}{#2}{#3}
    \else
      \ifx\@cftemp\cftsubsecname \@cftsetindents{subsec}{#2}{#3}
      \else
        \ifx\@cftemp\cftsubsubsecname \@cftsetindents{subsubsec}{#2}{#3}
        \else
          \ifx\@cftemp\cftparaname \@cftsetindents{para}{#2}{#3}
          \else
            \ifx\@cftemp\cftsubparaname \@cftsetindents{subpara}{#2}{#3}
            \else
              \ifx\@cftemp\cftfigname \@cftsetindents{fig}{#2}{#3}
              \else
                \ifx\@cftemp\cftsubfigname \@cftsetindents{subfig}{#2}{#3}
                \else
                  \ifx\@cftemp\cfttabname \@cftsetindents{tab}{#2}{#3}
                  \else
                    \ifx\@cftemp\cftsubtabname \@cftsetindents{subtab}{#2}{#3}
                    \else
                      \@cftsetindents{#1}{#2}{#3}
                    \fi
                  \fi
                \fi
              \fi
            \fi
          \fi
        \fi
      \fi
    \fi
  \fi
}

%    \end{macrocode}
% \end{macro}
%
% \begin{macro}{\@cftsetindents}
% |\@cftsetindents{|\meta{X}|}{|\meta{indent}|}{|\meta{numwidth}|}| 是 |\cftsetindents| 的
% 内部版本(internal version),在本例中 \meta{X}\ 是条目的内部(简写)名称。
%    \begin{macrocode}
\newcommand{\@cftsetindents}[3]{%
  \setlength{\@nameuse{cft#1indent}}{#2}
  \setlength{\@nameuse{cft#1numwidth}}{#3}
}

%    \end{macrocode}
% \end{macro}
%
%
% \subsection[切换页码]{\heiti 切换页码}
% \changes{v2.0}{2001/03/03}{Added page number switching off/on}
%
% \begin{macro}{\@cftpnumoff}
% |\@cftpnumoff{|\meta{shorthand}|}| 是关闭页码的主要工具。\meta{shorthand}\  参数
% 是条目的简写名称(shorthand name)(例如 |subsec| 表示 |subsection|)。
% 宏重新定义 |\cftXnumfill| 命令,这样就没有引线(leader),页码将被忽略。
%    \begin{macrocode}
\newcommand{\@cftpnumoff}[1]{%
  \@namedef{cft#1fillnum}##1{%
    \cftparfillskip\@nameuse{cft#1afterpnum}\par}}

%    \end{macrocode}
% \end{macro}
%
% \begin{macro}{\cftchapname}
% \begin{macro}{\cftsecname}
% \begin{macro}{\cftsubsecname}
% \begin{macro}{\cftsubsubsecname}
% \begin{macro}{\cftparaname}
% \begin{macro}{\cftsubparaname}
% \begin{macro}{\cftfigname}
% \begin{macro}{\cftsubfigname}
% \begin{macro}{\cfttabname}
% \begin{macro}{\cftsubtabname}
% 不幸的是,早期的设计决定(design decision)是使用 |sec| 这样的缩写(shorthands)来表示 |section|。
% 对于页面切换(page switching),我需要能够将缩写(shorthands)和普通写法(longhands)关联起来。
%    \begin{macrocode}
\newcommand*{\cftchapname}{chapter}
\newcommand*{\cftsecname}{section}
\newcommand*{\cftsubsecname}{subsection}
\newcommand*{\cftsubsubsecname}{subsubsection}
\newcommand*{\cftparaname}{paragraph}
\newcommand*{\cftsubparaname}{subparagraph}
\newcommand*{\cftfigname}{figure}
\newcommand*{\cftsubfigname}{subfigure}
\newcommand*{\cfttabname}{table}
\newcommand*{\cftsubtabname}{subtable}

%    \end{macrocode}
% \end{macro}
% \end{macro}
% \end{macro}
% \end{macro}
% \end{macro}
% \end{macro}
% \end{macro}
% \end{macro}
% \end{macro}
% \end{macro}
%
% \begin{macro}{\cftpagenumbersoff}
% 用于关闭页码的用户级命令是 |\cftpagenumbersoff{|\meta{entry}|}|,这里的 \meta{entry}\ 是普通写法(longhands)。
% 此宏的主要任务是确定 \meta{entry}\ 相应缩写名(shorthand name),然后调用 |\@cftpnumoff| 完成工作。
% 对于 |part| 和用户定义的条目(user-defined entries),普通写法和缩写的条目名称(entry names)相同。
%    \begin{macrocode}
\DeclareRobustCommand{\cftpagenumbersoff}[1]{%
  \def\@cftemp{#1}
  \ifx\@cftemp\cftchapname
    \@cftpnumoff{chap}
  \else
    \ifx\@cftemp\cftsecname \@cftpnumoff{sec}
    \else
      \ifx\@cftemp\cftsubsecname \@cftpnumoff{subsec}
      \else
        \ifx\@cftemp\cftsubsubsecname \@cftpnumoff{subsubsec}
        \else
          \ifx\@cftemp\cftparaname \@cftpnumoff{para}
          \else
            \ifx\@cftemp\cftsubparaname \@cftpnumoff{subpara}
            \else
              \ifx\@cftemp\cftfigname \@cftpnumoff{fig}
              \else
                \ifx\@cftemp\cftsubfigname \@cftpnumoff{subfig}
                \else
                  \ifx\@cftemp\cfttabname \@cftpnumoff{tab}
                  \else
                    \ifx\@cftemp\cftsubtabname \@cftpnumoff{subtab}
                    \else
                      \@cftpnumoff{#1}
                    \fi
                  \fi
                \fi
              \fi
            \fi
          \fi
        \fi
      \fi
    \fi
  \fi
}

%    \end{macrocode}
% \end{macro}
%
% \begin{macro}{\cftpagenumberson}
% |\cftpagenumberson{|\meta{entry}|}| 是用于反转相应的 |\cftpagenumbersoff| 的用户级命令。
%    \begin{macrocode}
\DeclareRobustCommand{\cftpagenumberson}[1]{%
  \def\@cftemp{#1}
  \ifx\@cftemp\cftchapname
    \@cftpnumon{chap}
  \else
    \ifx\@cftemp\cftsecname \@cftpnumon{sec}
    \else
      \ifx\@cftemp\cftsubsecname \@cftpnumon{subsec}
      \else
        \ifx\@cftemp\cftsubsubsecname \@cftpnumon{subsubsec}
        \else
          \ifx\@cftemp\cftparaname \@cftpnumon{para}
          \else
            \ifx\@cftemp\cftsubparaname \@cftpnumon{subpara}
            \else
              \ifx\@cftemp\cftfigname \@cftpnumon{fig}
              \else
                \ifx\@cftemp\cftsubfigname \@cftpnumon{subfig}
                \else
                  \ifx\@cftemp\cfttabname \@cftpnumon{tab}
                  \else
                    \ifx\@cftemp\cftsubtabname \@cftpnumon{subtab}
                    \else
                      \@cftpnumon{#1}
                    \fi
                  \fi
                \fi
              \fi
            \fi
          \fi
        \fi
      \fi
    \fi
  \fi
}

%    \end{macrocode}
% \end{macro}
%
%
% \begin{macro}{\@cftpnumon}
% |\@cftpnumon{|\meta{shorthand}|}| 是关闭页面编号的主要工具。\meta{shorthand}\ 参数
% 是条目(entry)的缩写名(shorthand name)(例如 |subsec| 表示 |subsection|)。宏定义 |\cftXnumfill| 命令
% 以对应于默认定义(default definition)。
%    \begin{macrocode}
\newcommand{\@cftpnumon}[1]{%
  \@namedef{cft#1fillnum}##1{%
    {\@nameuse{cft#1leader}}\nobreak
    \makebox[\@pnumwidth][\cftpnumalign]{
        \@nameuse{cft#1pagefont}##1}\@nameuse{cft#1afterpnum}\par
  }%
}
%    \end{macrocode}
% \end{macro}
%
%
%
%
%
%
% \subsection[实验性工具]{\heiti 实验性工具}
%
% 本节中的代码是实验性的(experimental),但从某种意义上来说,在未来这些功能(future)可能会被修改,
% 而不是代码无法工作。
%
% \begin{macro}{\cftchapterprecis}
% 这是实验性的(experimental)。在调用它的位置处 |\cftchapterprecis{|\meta{text}|}| 排版 \meta{text},
% 并将 \meta{text}\ 添加到 \file{.toc}\ 文件。期望在 |\chapter| 命令后立即调用它。
%    \begin{macrocode}
\newcommand{\cftchapterprecis}[1]{%
  \cftchapterprecishere{#1}
  \cftchapterprecistoc{#1}}
%    \end{macrocode}
% \end{macro}
%
% \begin{macro}{\cftchapterprecishere}
% |\cftchapterprecishere{|\meta{text}|}| 排版\meta{text}。期望在 |\chapter| 命令后立即调用它。
% 首先添加一些负垂直空间(negative vertical space),使其更靠近章标题(chapter heading)。
%    \begin{macrocode}
\newcommand{\cftchapterprecishere}[1]{%
  \vspace*{-2\baselineskip}
%    \end{macrocode}
% 在 |quote| 环境中使用斜体字体(italic font)排版其参数。
%    \begin{macrocode}
  \begin{quote}\textit{#1}\end{quote}}
%    \end{macrocode}
% \end{macro}
%
% \begin{macro}{\cftchapterprecistoc}
% |\cftchapterprecistoc{|\meta{text}|}| 将 \meta{text}\ 添加到 \file{.toc}\ 文件。
% \meta{text}\ 将使用斜体字体(italic font)在与 |\chapter| 标题(heading)的标题文本(title text)
% 相同的页边距(margins)内排版。
%    \begin{macrocode}
\newcommand{\cftchapterprecistoc}[1]{\addtocontents{toc}{%
%    \end{macrocode}
% 启动一个组(group)以对分段(paragraphing)进行局部更改(localize changes)。将左边距(left margin)
% 设置为章缩进(chapter indent)加上章编号宽度(chapter number width)。
%    \begin{macrocode}
  {\leftskip \cftchapindent\relax
   \advance\leftskip \cftchapnumwidth\relax
%    \end{macrocode}
% 将右侧边距设置为 |\@tocrmarg|。
%    \begin{macrocode}
   \rightskip \@tocrmarg\relax
%    \end{macrocode}
% 使用斜体字体(italic font)排版 \meta{text},然后确保段落结束(使用局部间距[local skips])。
% 最后结束组(group),我们就完成了。
%    \begin{macrocode}
   \textit{#1}\protect\par}}}

%    \end{macrocode}
% \end{macro}
%
% \begin{macro}{\cftlocalchange}
% |\cftmakelocalchange{|\meta{file}|}{|\meta{pnumwidth}|}{|\meta{tocrmarg}|}| 在 \meta{file}\ 中
% 创建一个条目(entry)来更改 |\@pnumwidth| 和 |\@tocrmarg| 的值。
%    \begin{macrocode}
\newcommand{\cftlocalchange}[3]{%
  \addtocontents{#1}{\protect\cftsetpnumwidth{#2} \protect\cftsetrmarg{#3}}}
%    \end{macrocode}
% \end{macro}
%
% \begin{macro}{\cftaddtitleline}
% |\cftaddtitleline{|\meta{file}|}{|\meta{kind}|}{|\meta{title}|}{|\meta{page}|}| 在 \meta{file}\ 中
% 添加一个 |\contentsline| 条目(entry)和给定的信息(given information)。
%    \begin{macrocode}
\newcommand{\cftaddtitleline}[4]{\addtocontents{#1}{%
  \protect\contentsline{#2}{#3}{#4}}}
%    \end{macrocode}
% \end{macro}
%
% \begin{macro}{\cftaddnumtitleline}
% |\cftaddtitleline{|\meta{file}|}{|\meta{kind}|}{|\meta{num}|}{|\meta{title}|}{|\meta{page}|}| 在 \meta{file}\ 中
% 添加一个 |\contentsline| 条目(entry)和给定的信息(given information)。
% \changes{v2.3c}{2003/09/26}{Removed \cs{ignorespaces} from \cs{cftaddnumtitleline}}
%    \begin{macrocode}
\newcommand{\cftaddnumtitleline}[5]{\addtocontents{#1}{%
    \protect\contentsline{#2}{\protect\numberline{#3}#4}{#5}}}
%    \end{macrocode}
% \end{macro}
%
% 而且,如果已使用了旧的 \Lpack{hyperref},我们必须修复(fix up)这两个宏。
% \changes{2003/09/26}{v2.3c}{Hyperref fix for \cs{cftaddtitleline} and
%                             \cs{cftaddnumtitleline}}
%    \begin{macrocode}
\AtBeginDocument{%
  \@ifpackageloaded{hyperref}{%
    \renewcommand{\cftaddtitleline}[4]{\addtocontents{#1}{%
      \protect\contentsline{#2}{#3}{#4}{\@currentHref}}}
    \renewcommand{\cftaddnumtitleline}[5]{\addtocontents{#1}{%
      \protect\contentsline{#2}{\protect\numberline{#3}#4}{#5}{\@currentHref}}}
  }{}
}

%    \end{macrocode}
%
%
% \begin{macro}{\@starttoc}
%
% 好了,下面是一个大致全面的列表(list),其中 \cs{@starttoc}\ 在 \TeX{} Live 2014 中被重新定义。
% \begin{itemize}
% \item amsart, amsbook, amsdtx, amsproc
% \item asect
% \item latx.ltx (当然)
% \item newfloat
% \item flowfram
% \item gmampulex (?), gmtypos
% \item hyperref, memhfixc
% \item ijmart
% \item scrartcl, scrbook, scrreprt
% \item scrwtfile
% \item tocbasic
% \item tocstyle
% \item memoir
% \item multitoc
% \item nccsect
% \item notoccite
% \item artikel3, boek3, rapport3,
% \item rerunfilecheck
% \item parskip
% \item pdfwin
% \item revtex4
% \item devanagari
% \end{itemize}
% 如果我们想在这里重新定义 \cs{@starttoc},这就很困难了。上面的许多包/类不会与 tocloft 一起使用;
% 另一方面,我们不想过分践踏(trample)他人的代码。
%
% 因此,对于我们自己的工作,至少在现在,让我们更加保守一些,如果它是标准的 \LaTeX{}\ 版本,那么
% 只需连接(hook into) \cs{starttoc}。
%    \begin{macrocode}
\def\@starttoc@latex@orig#1{%
 \begingroup
   \makeatletter
   \@input{\jobname.#1}%
   \if@filesw
     \expandafter\newwrite\csname tf@#1\endcsname
     \immediate\openout \csname tf@#1\endcsname \jobname.#1\relax
   \fi
  \@nobreakfalse
\endgroup}
\ifx\@starttoc\@starttoc@latex@orig
  \def\@starttoc#1{%
    \csname cft#1prehook\endcsname
    \begingroup
      \makeatletter
      \@input{\jobname.#1}%
      \if@filesw
        \expandafter\newwrite\csname tf@#1\endcsname
        \immediate\openout \csname tf@#1\endcsname \jobname.#1\relax
      \fi
      \@nobreakfalse
    \endgroup
    \csname cft#1posthook\endcsname}
  \newcommand\cfttocprehook{}
  \newcommand\cftlofprehook{}
  \newcommand\cftlotprehook{}
  \newcommand\cfttocposthook{}
  \newcommand\cftlofposthook{}
  \newcommand\cftlotposthook{}
\else
  \PackageWarning{tocloft}{
    \string\@starttoc\space has already been redefined; tocloft bailing out.
  }
\fi
%    \end{macrocode}
% \end{macro}
%
%
%
%
%    这个宏包结束。
%    \begin{macrocode}
%</usc>
%    \end{macrocode}
%
% \newpage
% \bibliographystyle{alpha}
%
% \begin{thebibliography}{GMS94}
%
% \bibitem[Coc95]{SUBFIGURE}
% Steven Douglas Cochran.
% \newblock \emph{{The subfigure package}}.
% \newblock March 1995.
% \newblock (Available from CTAN as file \texttt{subfigure.dtx})
%
% \bibitem[Dru99]{MINITOC}
% Jean-Pierre Drucbert.
% \newblock \emph{{The minitoc package}}.
% \newblock August 1999.
% \newblock (Available from CTAN in subdirectory \texttt{/minitoc})
%
% \bibitem[GMS94]{GOOSSENS94}
% Michel Goossens, Frank Mittelbach, and Alexander Samarin.
% \newblock {\em The LaTeX Companion}.
% \newblock Addison-Wesley Publishing Company, 1994.
%
% \bibitem[Lin97]{FNCYCHAP}
% Ulf~A. Lindgren.
% \newblock \emph{{FncyChap V1.11}}.
% \newblock April 1997.
% \newblock (Available from CTAN in subdirectory \texttt{/fncychap})
%
% \bibitem[Lin95]{FLOAT}
% Anselm Lingnau.
% \newblock \emph{{An Improved Environment for Floats}}.
% \newblock March 1995.
% \newblock (Available from CTAN in subdirectory \texttt{/float})
%
% \bibitem[Wil96a]{ALGORITHM}
% Peter Williams.
% \newblock \emph{{Algorithms}}.
% \newblock April 1996.
% \newblock (Available from CTAN in subdirectory \texttt{/algorithm})
%
% \bibitem[Wil96b]{PRW96i}
% Peter~R. Wilson.
% \newblock \emph{{LaTeX for standards: The LaTeX package files user manual}}.
% \newblock NIST Report NISTIR, June 1996.
%
% \bibitem[Wil00]{TOCBIBIND}
% Peter~R. Wilson.
% \newblock \emph{{The tocbibind package}}.
% \newblock March 2000.
% \newblock (Available from CTAN as file \texttt{tocbibind.dtx})
%
% \bibitem[Wil01]{CCAPTION}
% Peter~R. Wilson.
% \newblock \emph{{The ccaption package}}.
% \newblock March 2001.
% \newblock (Available from CTAN as file \texttt{ccaption.dtx})
%
% \end{thebibliography}
%
%
% \Finale
% \newpage
% \PrintIndex
%
\endinput

%% \CharacterTable
%%  {Upper-case    \A\B\C\D\E\F\G\H\I\J\K\L\M\N\O\P\Q\R\S\T\U\V\W\X\Y\Z
%%   Lower-case    \a\b\c\d\e\f\g\h\i\j\k\l\m\n\o\p\q\r\s\t\u\v\w\x\y\z
%%   Digits        \0\1\2\3\4\5\6\7\8\9
%%   Exclamation   \!     Double quote  \"     Hash (number) \#
%%   Dollar        \$     Percent       \%     Ampersand     \&
%%   Acute accent  \'     Left paren    \(     Right paren   \)
%%   Asterisk      \*     Plus          \+     Comma         \,
%%   Minus         \-     Point         \.     Solidus       \/
%%   Colon         \:     Semicolon     \;     Less than     \<
%%   Equals        \=     Greater than  \>     Question mark \?
%%   Commercial at \@     Left bracket  \[     Backslash     \\
%%   Right bracket \]     Circumflex    \^     Underscore    \_
%%   Grave accent  \`     Left brace    \{     Vertical bar  \|
%%   Right brace   \}     Tilde         \~}
